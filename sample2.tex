\documentclass[runningheads]{llncs}
%
%%%% PACKAGES %%%%

\usepackage[linesnumbered,ruled,lined,noend]{algorithm2e}
\usepackage{amsmath}
\usepackage{amsfonts}
\usepackage{enumitem}
\usepackage[hidelinks]{hyperref}
\AtEndPreamble{%
\usepackage[capitalise,noabbrev,nameinlink]{cleveref}
\crefname{footnote}{footnote}{footnotes}
\crefname{page}{page}{pages}
\crefname{line}{line}{lines}


\newlist{rquest}{enumerate}{1}
\setlist*[rquest]{label={Q\arabic{rquesti}:},ref={Q\arabic{rquesti}}}
\crefname{rquesti}{Research Question}{Research Questions}
\Crefname{rquesti}{Research Question}{Research Questions}
}%



\usepackage[normalem]{ulem}

\usepackage{enumitem}

\usepackage{float}

\usepackage[prefix,abbreviations]{glossaries-extra}

\usepackage{colortbl}

\usepackage{booktabs,tabularx,widetable}
\usepackage{multirow,makecell}
\usepackage{multirow}

\usepackage{pdflscape}
\usepackage{afterpage}

\usepackage{siunitx}

\usepackage{changepage}

\usepackage{listings}
\lstset{
  basicstyle=\ttfamily,
  columns=fullflexible,
}

\usepackage{tikz}
\usetikzlibrary{shapes.geometric,calc,backgrounds}
\usetikzlibrary{snakes,arrows,shapes,positioning}

\usepackage{caption}
\usepackage{subcaption}


% fix line references in cref
\makeatletter
\renewcommand{\nllabel}[1]
 {{\let\@currentlabel\algocf@currentlabel
  \let\@currentcounter\algocf@currentcounter
  \label{#1}}}%

\renewcommand{\algocf@nl@sethref}[1]{%
  \renewcommand{\theHAlgoLine}{\thealgocfproc.#1}%
  \hyper@refstepcounter{AlgoLine}%
  \gdef\algocf@currentlabel{#1}%
  \gdef\algocf@currentcounter{AlgoLine}%
 }%
\makeatother


\makeatletter
\newcommand{\@chapapp}{\relax}%
\makeatother
\usepackage[title,toc,titletoc,header]{appendix}


%%%% MACRO DEFINITIONS %%%%
\newcommand{\F}{\ensuremath{\mathbb{F}}}
\newcommand{\Fq}[1]{\ensuremath{\mathbb{F}_{q}^{#1}}}

\newcommand{\mat}[1]{\mathbf{#1}}

\newcommand{\mesg}{\ensuremath{\mathsf{msg}}\xspace}
\newcommand{\signedmessage}{\ensuremath{\mathsf{msg}_\mathsf{sig}}\xspace}

\newcommand{\lseed}{{\ell_\textsf{tree\_seed}}}
\newcommand{\lpubseed}{{\ell_\textsf{pub\_seed}}}
\newcommand{\lprivseed}{{\ell_\textsf{sec\_seed}}}
\newcommand{\lmu}{{\ell_{\Fq{m \times m}}}}
\newcommand{\lnu}{{\ell_{\Fq{n \times n}}}}
\newcommand{\lsalt}{{\ell_\textsf{salt}}}
\newcommand{\ldigest}{{\ell_\textsf{digest}}}

\newcommand{\mmperf}{\ensuremath{p_\text{mm}}\xspace}
\newcommand{\pipmperf}{\ensuremath{p_{\pi\text{pm}}}\xspace}
\newcommand{\pimmperf}{\ensuremath{p_{\pi\text{mm}}}\xspace}

\newcommand{\B}{\mathcal{B}}
\newcommand{\C}{\mathcal{C}}
\newcommand{\D}{\mathcal{D}}

\newcommand{\Ts}{\mathsf}

\newcommand{\typeA}{\Fq{m \times m}}
\newcommand{\typeB}{\Fq{n \times n}}
\newcommand{\typeAinv}{\operatorname{GL}_m(q)}
\newcommand{\typeBinv}{\operatorname{GL}_n(q)}
\newcommand{\typeG}{\Fq{k \times mn}}

\newcommand{\GLmq}{\operatorname{GL}_m(q)}
\newcommand{\GLnq}{\operatorname{GL}_n(q)}

\newcommand{\medspi}[3]{\pi_{#1, #2}(#3)}

\newcommand{\pk}{\ensuremath{\Ts{pk}}\xspace}
\newcommand{\sk}{\ensuremath{\Ts{sk}}\xspace}
\newcommand{\sig}{\textsf{sig}}

\newcommand{\ExpandSystMat}{\mathsf{ExpandSystMat}}
\newcommand{\ExpandInvMat}{\mathsf{ExpandInvMat}}
\newcommand{\ParseHash}{\mathsf{ParseHash}_{s,t,w}}
\newcommand{\compress}{\mathsf{Compress}}
\newcommand{\decompress}{\mathsf{Decompress}}
\newcommand{\randombytes}{\mathsf{Randombytes}}
\newcommand{\XOF}{\mathsf{XOF}}
\newcommand{\SeedTree}{\mathsf{SeedTree}}
\newcommand{\SeedTreeToPath}{\mathsf{SeedTreeToPath}}
\newcommand{\ToBytes}{\mathsf{ToBytes}}
\newcommand{\SF}{\mathsf{SF}}
\newcommand{\Hash}{\mathsf{H}}

\newcommand{\decompressG}{\mathsf{DecompressG}}
\newcommand{\lpath}{{\ell_\spath}}
\newcommand{\lsig}{{\ell_\textsf{sig}}}
\newcommand{\PathToSeedTree}{\mathsf{PathToSeedTree}}

\newcommand{\spath}{\textsf{path}}

\newcommand{\sw}{\ensuremath{S_1}\xspace}
\newcommand{\sx}{\ensuremath{S_2}\xspace}
\newcommand{\sy}{\ensuremath{S_3}\xspace}
\newcommand{\sz}{\ensuremath{S_4}\xspace}
\newcommand{\sm}{\ensuremath{S_{mat}}\xspace}

\SetKw{KwGoto}{goto}
\SetKw{KwOr}{or}



\newabbreviation{MEDS}{MEDS}{Matrix Equivalence Digital Signature}
\newabbreviation[prefixfirst={the~}]{NIST}{NIST}{National Institute of Standards and Technology}
\newabbreviation{PQC}{PQC}{Post-Quantum Cryptography}
\newabbreviation{MCE}{MCE}{Matrix Code Equivalence}
\newabbreviation{KEM}{KEM}{Key Encapsulation Mechanism}
\newabbreviation{BRAM}{BRAM}{Block RAM}
\newabbreviation{ALU}{ALU}{Arithmetic Logic Unit}
\newabbreviation{XOF}{XOF}{eXtendible Output Function}
\newabbreviation{TRNG}{TRNG}{True Random Number Generator}
\newabbreviation{PRNG}{PRNG}{Pseudo Random Number Generator}
\newabbreviation{KAT}{KAT}{Known Answer Test}
\newabbreviation{FS}{FS}{Fiat-Shamir}
\newabbreviation{RSA}{RSA}{Rivest–Shamir–Adleman}
\newabbreviation{DH}{DH}{Diffie-Hellman}
\newabbreviation{ECC}{ECC}{Elliptic Curve Cryptography}


%
\begin{document}
%
\title{Unified MEDS Accelerator}
%
%\titlerunning{Abbreviated paper title}

\author{%
  Sanjay Deshpande\inst{1}\orcidID{0009-0005-0938-9547}\and
  Yongseok Lee\inst{2}\orcidID{0000-0002-1608-809X}\and
  Mamuri Nawan\inst{3}\orcidID{0009-0004-1573-8718}\and 
  Kashif Nawaz\inst{3}\orcidID{0000-0002-3887-7364}\and
  Ruben Niederhagen\inst{4,5}\and
  Yunheung Paek\inst{2}\orcidID{0000-0002-6412-2926}\and
  Jakub Szefer\inst{6}\orcidID{0000-0001-9721-3640}
}

\authorrunning{Deshpande et al.}

\institute{%
  CASLAB, DEE, Yale University, New Haven, US, \email{sanjay.deshpande@yale.edu}
   \and
  ECE and ISRC, Seoul National University, Seoul, South Korea,\\ \email{yslee@sor.snu.ac.kr,ypaek@snu.ac.kr}
   \and
  CRC, Technology Innovation Institute, Abu Dhabi, UAE,\\ \email{mamuri@tii.ae, kashif.nawaz@tii.ae}
   \and
  IIS, Academia Sinica, Taipei, Taiwan, \email{ruben@polycephaly.org}
   \and
  IMADA, University of Southern Denmark, Odense, Denmark
  \and
  Northwestern University, Evanston, US, \email{jakub.szefer@northwestern.edu}
}

\maketitle

\begin{abstract}
  The Matrix Equivalence Digital Signature (MEDS) scheme,
a code-based candidate in the first round of NIST's
Post-Quantum Cryptography (PQC) standardization process,
offers competitively small signature sizes
but incurs high computational costs
for signing and verification.
This work explores how
a high-performance FPGA-based hardware implementation
can enhance MEDS performance
by leveraging the inherent parallelism of its computations,
while examining the trade-offs
between performance gains and resource costs.
This work in particular proposes a unified hardware architecture
capable of efficiently performing
both signing and verification operations 
within a single combined design.
The architecture jointly supports all security parameters,
including the dynamic, run-time handling of different prime fields
without the need to re-configure the FPGA.
This work also evaluates the resource overhead
of supporting different prime fields
in a single design,
which is relevant
not only for MEDS
but also for other cryptographic schemes
requiring similar flexibility. This work demonstrates that custom hardware
for PQC signature schemes can flexibly support different prime fields with limited resource overhead.
For example, for NIST security Level I,
our implementation achieves signing times of
\SI{4.5}{\milli\second} to \SI{65.2}{\milli\second}
and verification times of
\SI{4.2}{\milli\second} to \SI{64.5}{\milli\second}
utilizing 22k to 72k LUTs
and 66 to 273 DSPs
depending on design variant
and optimization goal.



  \enlargethispage{\baselineskip}

  \keywords{Post-Quantum Cryptography  \and Digital Signature Algorithm \and  Matrix Equivalence Digital Signature (MEDS)}
\end{abstract}


\section{Introduction}
\label{sec:intro}

\gls{MEDS} is a code-based digital signature scheme
based on the hardness assumption
of the \gls{MCE} problem.
It was submitted to the
\gls{NIST} \gls{PQC} Signature standardization process
launched in 2023,
but it did not advance to the second round
due to its comparatively low performance
(e.g., according to data from the PQM4 project\footnote{\url{https://github.com/mupq/pqm4}},
when evaluated in software
on an embedded 
evaluation board,
MEDS is 25$\times$, 244$\times$, 856$\times$ slower
in key generation, signing, and verification times respectively
when compared to NIST standard ML-DSA Dilithium \cite{cryptoeprint:2024/112}
on security level I)
and the novelty of its security assumptions.
Given the importance of ensuring
the practicality of the cryptographic schemes,
this paper focuses on developing
a unified high-performance hardware implementation
of the \gls{MEDS} signing and verification operation.
We present a joint design
that supports all security parameter sets,
selectable at runtime,
and evaluate the overhead
of accommodating arithmetic
for different finite fields
in a single design.
Our results on the overhead generalize well
and can be transferred to other schemes
with similar parameter set specifications
such as the NIST PQC Signature Round 2 submission CROSS \cite{NISTPQC-ADD-R2:CROSS24},
which has parameter sets for $\F_{127}$ and $\F_{509}$.
Other related schemes such as LESS \cite{NISTPQC-ADD-R2:LESS24} and PERK \cite{NISTPQC-ADD-R2:PERK24}
that currently are using a single small binary field
might benefit from investigating trade-offs
of offering parameter sets with different fields.

By addressing the computational challenges
associated with \gls{MEDS},
we aim to contribute to the broader goal
of making post-quantum cryptography
viable for widespread adoption
in the face of advancing quantum computing technologies.
To the best of our knowledge,
this work presents
the first hardware implementation of~\gls{MEDS}.
Our implementation operates in constant time
(i.e.,
there are no timing variations
depending on secret data),
though it does not include additional side-channel protections.

\paragraph{Related Work.}
%
Gaussian elimination is an important step in \gls{MEDS}
as well as many other \gls{PQC} schemes.
Early work~\cite{DBLP:journals/tc/HochetQR89}
used processor arrays as large as the matrix being processed.
The work in~\cite{DBLP:journals/tc/ShoufanWMHK10}
presented one of the first hardware accelerators
for the code-based Niederreiter cryptosystem.
The work used a systolic processor array
for performing Gaussian systemization.
Later work on Classic McEliece
also used a systemizer similar to \cite{DBLP:journals/tc/ShoufanWMHK10}
with smaller processor array.
For example,
in~\cite{DBLP:conf/reconfig/WangSN16,CHES:WanSzeNie17}
the authors present a hardware design
of a key-generation module for the Niederreiter cryptosystem.
In~\cite{TCHES:CCDLNSW22},
the authors implemented
a complete design of Classic McEliece
compliant with the specification
submitted to \pgls{NIST} standardization process
including binary field systemizers.
We are using a similar systemizer in our work.

Although there are similar computations
in regard to matrix systemization
between \gls{MEDS} and Classic McEliece,
compared to Classic McEliece,
instead of working with binary fields,
\gls{MEDS} requires $\F_q$ with prime $q$.
As a result,
the latencies
(in overall wall-clock time)
of finite field operations are larger
compared to when using $\F_2$ or a small binary extension filed $\F_q^{m}$.
This difference necessitates significant design effort
for the systemizer design,
e.g., a different control logic
and a more complex pipelining process.

Within current \gls{NIST} standardization process
for digital signatures schemes,
in \cite{DBLP:conf/cosade/SayariMAKS24,cryptoeprint:2023/1267}
the authors develop hardware implementations
of the MAYO digital signature scheme.
In~\cite{DBLP:journals/tches/DeshpandeHSY24},
authors present the first hardware implementation
of the SDitH digital signature scheme.
\cite{SP:dPRS23} describes hardware FPGA implementations
of the Raccoon digital signature scheme.


In~\cite{SAC:WJWDGSN19},
the authors present several hardware accelerators
that work with a RISC-V core
to accelerate the XMSS signature scheme.
In~\cite{TCHES:BCHKPSY23},
the authors present hardware implementation
of the Oil and Vinegar (OV) signature scheme.
In~\cite{tang2011high} authors presented an FPGA-based
implementation of the multivariate-based signature scheme Rainbow.
Later, in~\cite{ferozpuri2018high} authors presented high-speed, FPGA-based implementation of Rainbow
with updated parameters for NIST's first round of PQC standardization process.
Both designs
are operating on small binary fields
using a processor array
similar to the code-based designs
discussed above.
Among ASIC designs, in~\cite{zhu2023repqc} authors present a processor
designed specifically for PQC algorithms, which can support schemes such as
Saber, Kyber, Dilithium, NTRU, McEliece, and Rainbow.
This design also includes a module for the systemization
of matrices over small binary fields,
but instead of a processor array
composed of several vector units,
the authors here are using only a single a vector unit.

Hardware architectures for algorithms
already being standardized by \gls{NIST} include
\cite{DBLP:journals/iacr/SchmidAWZW23},
where the authors implement a hardware design
of the FALCON scheme
and
\cite{DBLP:conf/cardis/LandSG21,DBLP:conf/fpt/BeckwithNG21}
where the authors explore hardware implementations of CRYSTALS-Dilithium
(being standardized under the name ``ML-DSA'').
\cite{DBLP:conf/dsd/AmietLCZ20,DBLP:journals/iacr/Saarinen24,10.1145/3728469}
explores implementations of SPHINCS+
(being standardized under the name ``SLH-DSA'').

We are not aware of any other hardware implementation of \gls{MEDS}
and our work presents a hardware design of \gls{MEDS} that provides 
a competitive performance and resource utilization compared
hardware implementations of other signature schemes.

\paragraph{Contributions.}
%
This paper introduces
the first hardware implementation
of the sign and verify operations
for the \gls{PQC} signature scheme \gls{MEDS}.
Our implementation
combines both operations
into a single design
sharing most resources
between sign and verify.
We further
provide joint support
of all security parameter sets
in one single design
selectable at runtime.

We aim to address the following research questions:
%
\begin{rquest}
  \item Does \gls{MEDS} provide sufficient inherent parallelism
        to accelerate the sign and verify operations
        to a competitive level?
        What are the associated resource costs
        for achieving such a speed-up? \label{RQ:1}
  \item To what extent can resources be shared
        in a combined design for the sign and verify operations? \label{RQ:2}
  \item What is the overhead
        of supporting all security parameters at runtime,
        particularly concerning arithmetic
        in two distinct finite fields? \label{RQ:3}
\end{rquest}

The source code of our hardware designs is available under an open source license at \url{https://github.com/caslab-code/pqc-hw-meds}.


\section{Preliminaries}
\label{sec::prelim}

We first introduce our notation in \cref{sec::notation}
and then explain the relevant details
of the \gls{MEDS} specification
in \cref{sec::meds}.


\subsection{Notation}
\label{sec::notation}

We are following the notation
of the \gls{MEDS} specification document \cite{NISTPQC-ADD-R1:MEDS23}
and
denote matrices with bold capital letters,
e.g., $\mat{M}$, $\mat{A}$, and $\mat{B}$.
We follow the \gls{MEDS} specification in denoting submatrices
with square brackets,
e.g., $\mat{M}[a, b; c, d]$ denotes the submatrix
of the intersection of rows $a$ to $b$
and columns $c$ to $d$
of matrix $\mat{M}$ \cite[Section 2.1]{NISTPQC-ADD-R1:MEDS23}.
If no row or column range is provided,
all rows or columns are included.
%
$\F_q$ is a finite field with $q$ elements.
We define an
$[m \times n, k]$
matrix rank metric code over $\F_q$
as
$k$-dimensional subspace of $\Fq{m \times n}$.
Here, $m$ and $n$ are the codeword sizes
and $k$ the code~dimension.

The operation $\SF(\mat{M})$
returns the systematic form of a matrix $\mat{M}$
if it exists
or $\bot$ if not.
%
The operation $\medspi{\mat{A}}{\mat{B}}{\mat{M}}$
with $\mat{A} \in \Fq{m \times m}$,
$\mat{B} \in \Fq{n \times n}$,
and
$\mat{M} \in \Fq{k \times mn}$
first maps each row $i \in \{0, \dots, k-1\}$ of $\mat{M}$
to a matrix $\mat{P}_i \in \Fq{m \times n}$
such that $\mat{P}_i[\lfloor j/n \rfloor ; j \bmod n] = \mat{M}[i, j]$
for $j \in \{0, \dots, mn-1\}$.
It then computes $\mat{P}'_i \in  \Fq{m \times n}$
as $\mat{P}'_i = \mat{A}\mat{P}_i\mat{B}$,
maps the $\mat{P}'_i$
back to the rows $i$ of a matrix $\mat{M}' \in \Fq{k \times mn}$,
and
finally returns $\mat{M}'$ as result.


\subsection{MEDS}
\label{sec::meds}

The \gls{MEDS} scheme \cite{NISTPQC-ADD-R1:MEDS23,AFRICACRYPT:CNPRRST23}
was a submissions
to the on-ramp to the \gls{NIST} \gls{PQC} signature standardization process%
\footnote{\url{https://csrc.nist.gov/Projects/pqc-dig-sig/round-1-additional-signatures}}
but it did not advance to the second round.
It is based on the notion of \glsfirst{MCE}:
%
\begin{definition}[Matrix Code Equivalence]
  Let $\C$ and $D$ be two $[m \times n, k]$ matrix codes over~$\F_q$.
  We say that $C$ and $\D$ are \emph{equivalent}
  if there exist two invertible matrices $\mat{A} \in \Fq{m \times m}$
  and $\mat{B} \in \Fq{n \times n}$
  such that $\D = \mat{A} \cdot \C \cdot \mat{B} $,
  i.e., for all $\mat{C} \in \C,\ \mat{A} \cdot \mat{C} \cdot \mat{B} \in \D$.
\end{definition}

\noindent
This gives rise to the \gls{MCE} problem:
%
\begin{problem}[Matrix Code Equivalence Problem]
  Given two $k$-dimensional matrix codes $\C, \D \subset \Fq{m \times n}$,
  find two invertible matrices $\mat{A} \in \Fq{m \times m},\ \mat{B} \in \Fq{n \times n}$
  such that $\D = \mat{A} \cdot \C \cdot \mat{B}$.
\end{problem}

The \gls{MCE} problem
is at least as hard as the linear code equivalence problem
and as hard as the isomorphism of polynomials problem
\cite{EC:BouFouVeb13},
and as hard as the alternating trilinear form equivalence problem
\cite{DBLP:conf/stacs/GrochowQT21,EC:TDJPQS22}.
Please refer to \cite{NISTPQC-ADD-R1:MEDS23,AFRICACRYPT:CNPRRST23}
for a concrete security analysis.

\gls{MEDS} uses the \gls{MCE} problem
to construct a signature scheme
from an interactive $\Sigma$-protocol
using the \gls{FS} transform \cite{C:FiaSha86}
with $t$ rounds
and by applying some tricks
to improve signature size and performance,
i.e.,
by using $s$ matrix codes in the public key,
by using challenges with a fixed weight $w$,
by seeding 0-responses,
and by generating these seeds from a seed tree.

\paragraph{MEDS parameters.}
%
\Cref{tab::params} shows the parameter sets of \gls{MEDS}
from the first round
of the \gls{NIST} \gls{PQC} signature scheme standardization.
In May 2024,
the \gls{MEDS} submission team announced new parameter sets
in the NIST PQC mailing list%
\footnote{\url{https://groups.google.com/a/list.nist.gov/g/pqc-forum/c/pbT_DnPrc2A/m/ZPrIVSmFCQAJ} \label{fn::new_param}}
as reaction to refined attacks~\cite{DBLP:conf/eurocrypt/NarayananQT24}.
However,
they suggest to combine these new parameter sets
with signature-size optimization techniques
introduced in \cite{DBLP:conf/pqcrypto/ChouNRS24}.
These optimizations require algorithmic changes
compared to the \gls{MEDS} Round 1
specification document \cite{NISTPQC-ADD-R1:MEDS23}.
We decided to provide a hardware implementation
that follows the \gls{MEDS} Round 1 specification 
and therefore do not implement the optimization
from \cite{DBLP:conf/pqcrypto/ChouNRS24}
since there is no concrete specification
for that \gls{MEDS} variant
and since we are interested
in analyzing the effect
of supporting different prime fields
in a joint design.

\begin{table}[t] 
  \caption{\gls{MEDS} parameter sets.}
  \label{tab::params}

  \begin{tabularx}{\textwidth}
        {@{}
           l
           X
           S[table-format=4]
           S[table-format=2]
           S[table-format=2]
           S[table-format=2]
          @{\hskip 2.5em}
           S[table-format=1]
           S[table-format=4]
           S[table-format=2]
          @{\hskip 2.5em}
           S[table-format=6]
           S[table-format=6]
        @{}}
        \toprule
  {Level} & {Parameter Set} & {$q$} & {$n$} & {$m$} & {$k$} & {$s$} & {$t$} & {$w$} & {\makecell{$\pk$\\{\small(byte)}}} & {\makecell{$\sig$\\{\small(byte)}}} \\
  \midrule
   \multicolumn{10}{c}{MEDS Parameter Sets \cite[Table 2]{NISTPQC-ADD-R1:MEDS23}:}\\[2pt]
   Level I & MEDS-9923    &  4093 &  14 &  14 &  14 &   4 & 1152 &  14 &  9923 &  9896 \\
   Level I & MEDS-13220   &  4093 &  14 &  14 &  14 &   5 &  192 &  20 & 13220 & 12976 \\[2pt]
  \midrule
   Level III & MEDS-41711   &  4093 &  22 &  22 &  22 &   4 &  608 &  26 & 41711 & 41080 \\
   Level III & MEDS-69497   &  4093 &  22 &  22 &  22 &   5 &  160 &  36 & 55604 & 54736 \\[2pt]
  \midrule
   Level V & MEDS-134180  & 2039 &  30 &  30 &  30 &   5 &  192 &  52 & 134180 & 132528 \\
   Level V & MEDS-167717  & 2039 &  30 &  30 &  30 &   6 &  112 &  66 & 167717 & 165464 \\
  \bottomrule
   \\[-5pt]
   \multicolumn{10}{c}{New Parameter Sets May 2024 (signature bytes with seed tree)$^\text{\ref{fn::new_param}}$:}\\[2pt]
   Level I   &      & 4093 &  26 &  25 &  25 &   2 &  144 &  48 &  21595 &  5456  \\
   Level III &      & 4093 &  35 &  34 &  34 &   2 &  208 &  75 &  55520 & 10786  \\
   Level V   &      & 4093 &  45 &  44 &  44 &   2 &  272 & 103 & 122000 & 21052  \\
  \bottomrule
  \end{tabularx}
\end{table}

\paragraph{\gls{MEDS} key generation.}
%
To generate a key pair,
first chose a random $\mat{G}_0 \in \Fq{k \times mn}$.
Then choose random invertible matrices
$\mat{A}_1, \dots, \mat{A}_{s-1} \in \Fq{m \times m}$
and
$\mat{B}_1, \dots, \mat{B}_{s-1} \in \Fq{n \times n}$.
Finally, compute the matrices
$\mat{G}_1, \dots, \mat{G}_{s-1} \in \Fq{k \times mn}$
as
$\mat{G}_i = \SF(\medspi{\mat{A}_i}{\mat{B}_i}{\mat{G}_0}),\ i \in \{1, \dots, s-1\}$.
The public key of \gls{MEDS} consists of
the seed for randomly generating $\mat{G}_0 \in \Fq{k \times mn}$
and
$s-1$ matrices
$\mat{G}_1, \dots, \mat{G}_{s-1} \in \Fq{k \times mn}$.
The secret key of $s-1$ pairs of invertible matrices
$(\mat{A}^{-1}_1, \mat{B}^{-1}_1), \dots,
(\mat{A}^{-1}_{s-1}, \mat{B}^{-1}_{s-1})
\in \Fq{m \times m} \times \Fq{n \times n}$.

\paragraph{\gls{MEDS} signing.}
%
During signing,
the signer commits to $t$ matrices $\tilde{\mat{G}}_i \in \Fq{k \times mn},\ i \in \{0, \dots, t-1\}$
using $t$ maps of pairs of random invertible matrices
$(\tilde{\mat{A}}_0, \tilde{\mat{B}}_0), \dots,$ 
$(\tilde{\mat{A}}_{t-1}, \tilde{\mat{B}}_{t-1}) \in \Fq{m \times m} \times \Fq{n \times n}$
such that
$\tilde{\mat{G}}_i = \SF(\medspi{\tilde{\mat{A}}_i}{\tilde{\mat{B}}_i}{\tilde{\mat{G}}_0}),\ i \in \{0, \dots, t-1\}$.
The signer then hashes the $\tilde{\mat{G}}_i$ to the commitment hash $d$
and parses $d$ to the challenge vector $h_0, \dots, h_{t-1}$
of weight $w$
with $h_i \in \{0, s-1\}$.
The signature finally is composed by providing~$d$
as well as the seeds
$\tilde{\mat{A}}_i$ and $\tilde{\mat{B}}_i$ for all $i$ where $h_i = 0$
and
$\mu_i = \tilde{\mat{A}}_i \cdot \mat{A}_{h_i}^{-1}$ and
$\nu_i = \mat{A}_{h_i}^{-1} \cdot \tilde{\mat{B}}_i$
for all $i$ where $h_i > 0$.

\paragraph{\gls{MEDS} verification.}
%
For verification,
the verifier parses the challenge vector $h_0, \dots, h_{t-1}$ from $d$.
The verifier then computes $\hat{\mat{G}}_i \in \Fq{k \times mn},\ i \in \{0, \dots, t-1\}$
as
$\hat{\mat{G}}_i = \SF(\medspi{\mu_i}{\nu_i}{\mat{G}_{h_i}})$,
where
the matrices $\mu_i = \tilde{\mat{A}}_i$
and $\nu_i = \tilde{\mat{B}}_i$
are regenerated from the seed-tree seeds
for all $i$ with $h_i = 0$,
and
$\mu_i = \tilde{\mat{A}}_i \cdot \mat{A}_{h_i}^{-1}$ and
$\nu_i = \mat{A}_{h_i}^{-1} \cdot \tilde{\mat{B}}_i$
are parsed from the signature
for all $i$ with $h_i > 0$.
If the hash of the $\hat{\mat{G}}_i$
equals the commitment hash $d$,
the verification is successful.

\paragraph{Cost.}
%
The most expensive sub-operations in \gls{MEDS} are
$\SF$ with a cost of $\text{O}(mnk^2)$ finite field operations
for performing Gaussian elimination
on a matrix of size $k \times mn$
and
$\pi$ with a cost of $\text{O}(k mn^2 + k m^2n)$
for performing $k$ matrix products
of sizes $\Fq{m \times m}$ times $\Fq{m \times n}$ times $\Fq{n \times n}$.
Hence,
the computational cost of $\SF$ and $\pi$
is quartic in the security parameters with $k \approx m \approx n$.

For key generation,
we need to compute $\SF$ and $\pi$ each only $s$ times,
where $s \in \{4, 5, 6\}$
---
but for signing and verification,
we need to perform these costly operations
$t$ times
with $t \in \{112, 160, 192, 608, 1152\}$.
Therefore,
signing and verification are
$18.6$ to $288$ times more expensive
than key generation
and hence benefit the most from hardware acceleration.
Since verification performs almost the same computations
as signing
with only little difference in the control flow,
we combine both operations in a single design,
sharing as many resources between these operations as possible.


\section{Unified \gls{MEDS} Design}
\label{sec::design}

We call our hardware design of \gls{MEDS}
a \emph{unified} design, since it
\emph{combines} both the \gls{MEDS} signing and verification operations
into a single design,
and it provides
\emph{joint} support for all parameter sets
selectable at runtime,
while sharing hardware resources
between both operations and all parameter sets
as much as possible.

\Cref{alg:sign} and \cref{alg:verify}
in the Appendix
show the signing
and verification
operations in detail
\cite[Algorithms 11 and 12]{NISTPQC-ADD-R1:MEDS23}.
The inputs to the signing operation
are the secret key \sk
and the message \mesg
that needs to be signed.
The inputs to the verify operation
are the public key \pk
and the signed message
to be verified.
%
Comparing the main loop in
\cref{alg::sign::comm_start} to \cref{alg::sign::comm_end}
in \cref{alg:sign}
with the main loop in
\cref{alg::verify::comm_start} to \cref{alg::verify::comm_end}
in \cref{alg:verify}
shows that the main operations
of obtaining invertible matrices,
applying the $\pi$ and $\SF$ operations,
and computing a commitment hash
are structurally identical.
Hence, in our combined design,
resources for performing these computations
can be shared efficiently
between the sign and the verify operation
with little control~overhead.

\subsection{Top-down Overview of Our \gls{MEDS} Hardware Design}
\label{sec::HW-overview}

\begin{figure}[t]
  \centering
  \tikzstyle{sign} = [color=red,thick,-latex]
\tikzstyle{verify} = [color=blue,thick,-latex]
\tikzstyle{both} = [-latex]

\definecolor{colorbg1}{HTML}{DCEAF7}
\definecolor{colorbg2}{HTML}{FBE3D6}
\definecolor{colorbg3}{HTML}{F2CFEE}

\resizebox{\textwidth}{!}{%
\begin{tikzpicture}[align=center,node distance=1cm and 5mm, every node/.style={draw,fill=white, minimum width=2cm, minimum height=8mm}]

  \node (start) at (-1.5,0) [draw=none] {\bf start};

  \node (xof1) [right = of start, yshift=1cm] {\textsf{XOF} $\rho, \alpha$};
  \node (ST) [right = of xof1, xshift=5mm] {\textsf{SeedTree}};
  \node (expG0) [right = of xof1, yshift=-1cm, xshift=5mm] {$\ExpandSystMat\ \mat{G}_0$};

  \node (parseV) [right = of start, yshift=-1cm] {\textsf{ParseHash}};
  \node (path2ST) [right = of parseV, xshift=5mm] {\textsf{PathToSeedTree}};

  \node (xof2) at (-4,-3)  {\textsf{XOF} $\sigma_{\mat{\tilde{A}}_i}, \sigma_{\mat{\tilde{B}}_i}$};
  \node (expInvA) [fill=colorbg2, right = of xof2] {$\ExpandInvMat\ \mat{\tilde{A}}_i / \mu_i$};
  \node (expInvB) [fill=colorbg2, right = of expInvA] {$\ExpandInvMat\ \mat{\tilde{B}}_i / \nu_i$};
  \node (pi) [fill=colorbg3, right = of expInvB] {$\pi\ \mat{\tilde{G}}_i$};
  \node (SF) [fill=colorbg2, right = of pi] {$\SF\ \mat{\tilde{G}}_i$};
  \node (hash) [right = of SF] {$\Hash(\compress(\mat{\tilde{G}}_i))$};

  \node (comp) at (1,-7.5)  {check $d = d'$};

  \node (parse) at (1,-6)  {\textsf{ParseHash}};
  \node (mu) [fill=colorbg2, right = of parse, yshift=0.5cm, xshift=5mm] {$\mat{\tilde{A}}_i \mat{A}^{-1}_{h_i}$};
  \node (nu) [fill=colorbg2, right = of mu] {$\mat{B}^{-1}_{h_i} \mat{\tilde{B}}_i$};
  \node (ST2path) [right = of parse, yshift=-0.5cm, xshift=5mm] {\textsf{SeedTreeToPath}};

  \node (done) [right = of parse, xshift=6.25cm, draw=none] {\bf done};

  \draw[sign] (start) -- (xof1.west);
  \draw[sign] (xof1.east) -- (ST.west);
  \draw[sign] (ST.east) -| ([xshift=2.5mm]expG0.east);

  \draw[verify] (start) -- (parseV.west);
  \draw[verify] (parseV.east) -- (path2ST.west);
  \draw[verify] (path2ST.east) -| ([xshift=2.5mm]expG0.east);

  \draw[both] (start.east) -- (expG0.west);
  \draw[both] (expG0.east) -- ++(2,0) -- ++(0,-1.75) -| ([xshift=-7.5mm]xof2.west) -- (xof2.west);

  \draw[both] (xof2.east) -- (expInvA.west);
  \draw[both] (expInvA.east) -- (expInvB.west);
  \draw[both] (expInvB.east) -- (pi.west);
  \draw[both] (pi.east) -- (SF.west);
  \draw[both] (SF.east) -- (hash.west);

  \draw[both] ([xshift=2.5mm]hash.east) -- ++(0,7.5mm) coordinate(mark1) -| ([xshift=-5mm]xof2.west) coordinate[pos=0.5] (mark2);
  \node[draw=none, fill=colorbg1, minimum height=4mm] at ($(mark1)!0.5!(mark2)$) {$0 \leq i \leq t-1$};

  \draw[verify] ([xshift=-2.5mm]xof2.west) -- ++(0,-7mm) coordinate (mark1) -| ([xshift=2.5mm]xof2.east) coordinate[pos=0.5] (mark2);
  \node[draw=none, fill=colorbg1, minimum height=4mm] at ($(mark1)!0.5!(mark2)$) {$h_i > 0$};


\begin{scope}[on background layer]
  \draw[dashed, rounded corners, fill=colorbg1] ([xshift=-6.25mm,yshift=7mm]xof2.north west)
                                      rectangle ([xshift=3.75mm,yshift=-6.5mm]hash.south east);
\end{scope}

  \draw (hash.east) -- ++(0.5,0) -- ++(0,-1.25) -| ([draw=none,xshift=-5mm]parse.west) coordinate[pos=0.5] (mark);

  \draw[verify] (mark) -- ++(-0.25,0) |- (comp.west);

  \draw[sign] (mark) |- (parse.west);

  \draw[sign] (parse.east) -| ([xshift=-5mm]mu.west) -- (mu.west);
  \draw[sign] (parse.east) -| ([xshift=-5mm]ST2path.west) -- (ST2path.west);
  \draw[sign] (mu.east) -- (nu.west);
  \draw[sign] (ST2path.east) -| ([xshift=5mm]nu.east);

  \draw[sign] ([xshift=2.5mm]nu.east) -- ++(0,7.5mm) coordinate(mark1) -| ([xshift=-2.5mm]mu.west) coordinate[pos=0.5] (mark2);
  \node[draw=none, fill=white, minimum height=4mm] at ($(mark1)!0.5!(mark2)$) {$0 \leq i \leq t-1$};

  \draw[sign] (nu.east) -| ([xshift=-5mm,yshift=1mm]done.west) -- ([yshift=1mm]done.west);

  \draw[verify] (comp.east) -| ([xshift=-5mm,yshift=-1mm]done.west) -- ([yshift=-1mm]done.west);

\end{tikzpicture}
}%



  \caption{Overview of the operational flow of our combined sign and verify module,
           red lines for sign only, blue lines for verify only.}
  \label{fig:operational_flow}
\end{figure}

\Cref{fig:operational_flow}
shows the operation flow of our combined hardware implementation.
Data flow unique to the sign operation
is shown in red
and unique to the verify operation in blue.
First, both operations
expand $\mat{G}_0$
from seed $\sigma_{\mat{G}_0}$,
which is parsed from $\sk$ or $\pk$ respectively,
using the module \texttt{ExpandSystMat}.

Signature generation starts with
the generation
of the seed $\delta$ of length $\lprivseed$
from a randomness source.
We assume that the seed $\delta$ will be initialized
by a top-level hardware module,
e.g., by interfacing to a \gls{TRNG}
or by implementing the \gls{NIST} \gls{KAT} \gls{PRNG}.
The seed $\delta$ is then expanded
using the module \texttt{XOF}
to the root seed $\rho$ of length $\lseed$
of the seed tree
and a salt $\alpha$ of length $\lsalt$.
In our hardware design,
we accomplish this
again using a \texttt{SHAKE256} module
via a \texttt{XOF} wrapper.
Then the next step is the seed tree generation
using the root seed $\rho$
and the salt $\alpha$
with the \texttt{SeedTree} module.
Verification instead parses the hash
and then generates the partial seed tree
from the signature.
We implement this
in the modules \texttt{ParseHash}
and \texttt{PathToSeedTree}.

For the main loop
in sign and verify
we need to expand the seeds $\sigma_i$
from the seed tree leaves
into seeds $\sigma_{\mat{\tilde{A}}_i}$ and $\sigma_{\mat{\tilde{B}}_i}$
and expand these seeds
to $\mat{\tilde{A}}_i$ and $\mat{\tilde{B}}_i$
(i.e., $\mu_i$ and $\nu_i$ respectively for verify)
using a \texttt{ExpandInvMat} module.
For signing, we store each $\sigma_i$ in a memory which is later used to 
recompute $\sigma_{\mat{\tilde{A}}_i}$ and $\sigma_{\mat{\tilde{B}}_i}$ and then expand these seeds
to $\mat{\tilde{A}}_i$ and $\mat{\tilde{B}}_i$ for computation of $\mu_i$ and~$\nu_i$.
We then
feed the matrices into a \texttt{Pi} module
together with the matrix $\mat{G}_0$,
compute the systematic form $\mat{\tilde{G}}_i$ of the result
using the module \texttt{SF}.
We pull the absorb-function
of the hash function $\Hash$
into the loop
to avoid the need to store all $\mat{\tilde{G}}_i$.
We will describe the implementation of this loop in detail
in \cref{sec:signing}.

The steps after the main loop
are different for sign and verify.
For sign,
we also absorb the message \mesg
into the hash state
and finalize $\Hash$
to obtain the challenge vector.
Then,
to compute the final loop, we recompute $\sigma_{\mat{\tilde{A}}_i}$ and $\sigma_{\mat{\tilde{B}}_i}$ using $\sigma_i$, 
and then expand them to generate
$\mat{\tilde{A}}_i$ and $\mat{\tilde{B}}_i$
for the computation of $\mu_i$ and $\nu_i$
using a matrix-matrix multiplication module.
Eventually, we compute the
seed tree path for the signature
using \texttt{SeedTreeToPath}.
For verify, we only need to check
that $d = d'$
and set the return value of the module correspondingly.


\subsection{Joint Security Parameter Design}

One of the main questions we want to investigate in this work
is the hardware overhead of supporting several prime fields
for different security parameter sets
selectable at runtime
in one joint hardware design.
For modules
that require both 11-bit arithmetic in $\F_{2039}$ and 
12-bit arithmetic in  $\F_{4093}$,
we are using an input signal \lstinline{i_sel_4093}
that is pulled high for $\F_{4093}$
and low for $\F_{2039}$.
The bus width
for field elements
is set to \SI{12}{\bit} for the joint design,
with the most significant bit set to zero
when $\F_{2039}$ is being selected.
Therefore,
the cost for memory storing field elements
is defined by the larger field in the joint design.
A similar design approach of constructing field arithmetic modules 
capable of operating over two distinct prime fields was adopted in \cite{9946370,10483343}, 
where the authors integrated CRYSTALS-Kyber (which operates over a 12-bit prime field polynomial coefficients) 
and CRYSTALS-Dilithium (which utilizes a 23-bit prime field polynomial coefficients). 
In contrast, MEDS features prime fields that differ by only a single bit, 
enabling support for multiple fields with minimal overhead. 


If the Verilog code is synthesised
for one specific field only,
we are fixing \lstinline{i_sel_4093}
to constant $0$ or $1$ respectively
so that the optimization step during synthesis
can remove unnecessary logic automatically.
This enables us
to get precised resource estimates
for hardware designs supporting only one respective field
and for the joint design
with otherwise identical logic.

The remaining security parameters
besides the prime field
are matrix dimensions $n$, $m$, and $k$,
the number $s$ of $\pk$ matrices,
as well as $t$ and $s$ for the challenge vector.
All these other parameters
mostly affect the upper limit of loops
and can easily be selected
using multiplexers
at runtime
at little resource overhead.
We pre-compute related pre-defined constant values
(e.g., counter widths)
and provide them as macros and module parameters
during synthesis (and simulation).


\section{Implementation}
\label{sec::impl}

For the description of our implementation
of the \gls{MEDS} signature scheme,
we take a bottom-up approach.
The different building blocks
and sub-modules
are described
in the following
including finite field arithmetic,
matrix operations,
and the complete sign and verify module.

\subsection{Finite Field Arithmetic} \label{sec:finite_field_arithmetic}
%
As described in specified in \cref{sec::meds},
all underlying arithmetic operations
multiplication, addition, and inversion
in the \gls{MEDS} signature scheme
are performed in the prime field $\mathbb{F}_q$
where $q$ is either 
the 11-bit prime $2039 = 2^{11} - 9$
or
the 12-bit prime $4093 = 2^{12} - 3$.
We implement these operations
as integer arithmetic
modulo the prime $q$.
An overview of the resource consumption
and performance
for each module
is provided in \cref{tab:arith}.

\begin{table}[t]
  \centering
  \caption{Resource consumption and performance
           of the $\F_q$ arithmetic functions
           targeting an Xilinx Artix 7 (\texttt{xc7a200t}) FPGA.
  }
  \label{tab:arith}
  {
    \fontsize{8pt}{10pt}\selectfont
    {%\setlength{\tabcolsep}{1.2ex} 
\sisetup{table-auto-round,group-minimum-digits=4,group-digits=integer}
\begin{widetable}{\textwidth}{@{}
                                c
                                S[table-format=4]
                                S[table-format=2]
                                S[table-format=1]
                                S[table-format=2]
                                S[table-format=1.1]
                                S[table-format=1]
                                S[table-format=3]
                                S[table-format=1.3]
                                S[fixed-exponent=3, scientific-notation=fixed, table-format=2.3e1,
                                   round-precision=1, round-mode=figures]
                             @{}}
   \toprule
%Prime Field Operation	Area				F	Clock Cycles	Time	Time x Area
%    LUT    DSP    FF    BRAM    (MHz)        (us)    (x10^3)
   & &\multicolumn{4}{c}{\textbf{Resources}} \\
   \cmidrule{3-6}
   \textbf{Operation} & {$q$} & \multicolumn{2}{c}{\textbf{Area}} & \multicolumn{2}{c}{\textbf{Memory}} & {\bf Cycles} & {\bf Freq.} & {\bf Time} & {\bf Time$\times$Area}\\
   \cmidrule(r){3-4} 
   \cmidrule(l){5-6}
   & & {(LUT)} &{(DSP)} & {(FF)} & {(BRAM)} & {(cyc.)} & {(MHz)}  & {(us)} &  \\   
   \toprule
%\multicolumn{10}{c}{\bf Addition:}\\[1mm]
               & 2039   & 27    & 0    & 23    & {---} & 1    & 244       & 0.004091    & 110.457 \\
{\bf Addition} & 4093   & 28    & 0    & 25    & {---} & 1    & 221       & 0.004532    & 126.896 \\
               & {both} & 33    & 0    & 25    & {---} & 1    & 201       & 0.004982    & 164.406 \\
\midrule
%\multicolumn{10}{c}{\bf Multiplication:}\\[1mm]
                     & 2039   & 30    & 1    & 40    & {---} & 4    & 360.75    & 0.011088    & 332.64 \\
{\bf Multiplication} & 4093   & 44    & 1    & 41    & {---} & 4    & 356.50    & 0.011220    & 493.68 \\
                     & {both} & 46    & 1    & 45    & {---} & 4    & 343.05    & 0.011660    & 536.36 \\
\midrule
% \multicolumn{9}{c}{\bf Multiply-and-Add:}\\[1mm]
% $\F_{2039}$    &     &     &     &     &     &     &     &  \\
% $\F_{4093}$    &     &     &     &     &     &     &     &  \\
% Joint          &     &     &     &     &     &     &     &  \\
% \midrule
%\multicolumn{10}{c}{\bf Inversion (2x in parallel):}\\[1mm]
 \multirow{3}{*}{\makecell{\bf Inverse \\ $2 \times$ parallel}}
   & 2039   &  0     & 0    & 1     & 1     & 1    & 392    & 0.00255       & {---} \\
   & 4093   &  0     & 0    & 1     & 1.5   & 1    & 392    & 0.00255       & {---} \\
   & {both} & 12     & 0    & 1     & 2.5   & 1    & 269    & 0.003715      & 44.58 \\
  \bottomrule
\end{widetable}}



  }
\end{table}


\paragraph{Modular Multiplication.} \label{sec:modular_multiplication}
%
Since the size of each element in $\mathbb{F}_q$ is 11 or 12 bits,
we can use the DSPs available on the target FPGA (AMD Artix 7 {\tt xc7a200t-3})
to perform integer multiplication
followed by full reduction.
Using the DSPs helps to avoid long critical paths
in the FPGA design.
For modular reduction after integer multiplication,
we design a specific modular reduction unit targeting the $q$ value. 
As $q$ is in a pseudo-Mersenne form
($q = 2^{11}-9 = 2039$ or $q = 2^{12}-3 = 4093$),
we use the folding technique
to perform the modular reduction~\cite{DBLP:conf/arith/HasenplaughGG07}.
As the size of the multiplied output is up to $2 \lceil \log_2(q) \rceil$ bits
(i.e., 22 resp.~24 bits), 
we take the most-significant $\lceil \log_2(q) \rceil$ bits
(i.e., 11 resp.~12 bits), 
of the output,
multiply them by $2^{\lceil \log_2(q) \rceil} - q$
(i.e., $9 = 1001_b$ or $3 = 11_b$
---
since the binary representation of the factor is sparse,
we simple shift-and-add instead of multiplying),
and add this to the least significant~$\lceil \log_2(q) \rceil$ bits
(i.e., 11 resp.~12 bits).
As the result of the addition may have more than $\lceil \log_2(q) \rceil$ bits,
we repeat this folding process
until all most-significant bits are folded
and until there is no more carry generated
from addition after folding.
Eventually,
we use a multiplexer at the intermediate folded result $r$
to identify corner cases $r \geq q$
and in case we detect one of these cases,
we subtract the modulus $q$ from~$r$.

Many computations in the joint design
are the same for both fields.
The only difference in the folding reduction step is
that for $q = 2039$ we need to multiply the top bits by $9$
and for $q = 4093$ by $3$,
which only requires an additional multiplexer. 
Also the final conditional subtraction
just requires a multiplexer
to select the prime.

The input-independent latency
for our modular multiplication unit
for all variants
is four clock cycles.
One register is placed after the multiplication unit
followed by two registers
to pipeline the reduction unit efficiently
to reduce the overall critical path.
The output is not registered by default.


\paragraph{Modular Inversion.}
%
This operation is mostly required
for matrix systemization
(described in detail in \cref{sec:systemizer}).
To perform that computation efficiently,
we require low-latency field inversion.
To achieve this and
since the size of the finite field $\mathbb{F}_q$
with 11 and 12 bits is relatively small,
we decided not to use expensive, high-latency inversion algorithms
like the extended Euclidean algorithm,
Fermat's little theorem
or Montgomery inversion
for our modular inversion module.
Instead,
we trade computation for memory
and precompute the inverse of all elements in $\mathbb{F}_q$ at synthesis time
and store them as look-up table in \gls{BRAM}.
Hence,
we have a input-independent latency
of only one clock-cycle
for performing modular inversion. 
To use the memory resources as efficiently as possible,
we are exploiting the dual-ported interface of the \gls{BRAM}
to perform two independent inversions in parallel,
hence halving the effective resource cost.

The joint design
for this operation
provides look-up tables
for both primes
and some additional logic
for selecting the output
of the requested field.
The resource requirements are shown on \cref{tab:arith}.

\paragraph{Summary.}
%
The resource evaluation in \cref{tab:arith} demonstrates
that hardware support for multiple fields
imposes only a small additional cost,
provided the fields are ``related,''
meaning they share the same reduction algorithm. 
Similar works that support two distinct prime fields \cite{9946370,10483343} do not present a fine-grained analysis of the overhead introduced at the modular arithmetic level due to dual-field support; instead, they primarily emphasize the associated overhead at the level of polynomial multiplication units.
In our reduction module, when employing the folding technique,
efficient resource sharing can be achieved
under the condition
that the primes have a low Hamming distance.
Consequently,
supporting different prime fields in hardware
is not prohibitively resource-intensive
if the fields are selected carefully.
This can make using different prime fields
a viable design choice for cryptographic primitives
when using different fields offers meaningful benefits,
such as reduced signature or ciphertext sizes. 


\subsection{Matrix Multiplication} \label{sec:matrix_multiplication}
Matrix multiplication
is one of the most used operations
in the \gls{MEDS}
It is used in the $\pi$ operation
and for computing the $\mu_i$ and $\nu_i$
during signing.
Consequently,
we take care to design
an efficient matrix multiplication unit
that is parameterizable
at synthesis time
to enable performance trade-offs.
As mentioned in \cref{sec::HW-overview},
matrices in \gls{MEDS} are specified
to be sampled in row-wise order.

The general task of this module
is to compute $\mat{C} = \mat{A} \cdot \mat{B}$.
Instead of traditional schoolbook
matrix multiplication
with limited parallelism,
we are using a vector-based approach
similar to standard SIMD techniques
in our matrix multiplication design.

Asymptotically more efficient
matrix multiplication algorithms,
e.g., Strassen
or Coppersmith–Winograd,
do not apply to \gls{MEDS}
due to the relatively small matrix dimensions. 
Additionally, techniques such as the
trick by Arlazarov, Dinic, Kronrod, and Faradžev \cite{ArlazarovEtAl1970},
which are not asymptotically faster 
and are most effective for small fields
when the matrix dimension exceeds the field size
(e.g., $\mathrm{GF}(16)$ with matrix dimension up to~$133$
in~\cite{10.1007/978-3-031-54409-5_10}),
are also not applicable in the context of MEDS
due to the relatively large fields $\F_{2039}$ and $\F_{4093}$
and small matrix dimension of up to~$30$.

This approach allows us
to take advantage of the pipelining
of the $\F_q$ multiplier
and also allows us
to perform single-element additions
in each iteration.
This design also allows us
to stream the inputs to the multiplication unit
and to make best use of all pipelining registers.
The latency $l_\text{mm}$ of our matrix multiplication unit
can be computed
from the matrix dimensions $d_1$, $d_2$, and $d_3$,
the number of pipeline stages $l_\text{mmpipe}$,
and the performance parameter \mmperf
(i.e.,~the column-block width)
as $l_\text{mm} = l_\text{mmpipe} + \frac{d_1 \cdot d_2 \cdot d_3}{\mmperf}$.

\begin{figure}[t]
  \centering
  \begin{subfigure}[b]{0.45\textwidth}
    \centering
    \includegraphics[width=1.0\textwidth]{figs/matrix_multiplication.pdf}
    \caption{matrix multiplication module.}
    \label{fig::mat_mul}
  \end{subfigure}
  \begin{subfigure}[b]{0.5\textwidth}
    \centering
    \includegraphics[width=1.0\textwidth]{figs/pi.pdf}
    \caption{$\pi$ module.}
    \label{fig::pi_mod}
  \end{subfigure}
  \caption{Hardware design of the matrix multiplication and $\pi$ modules.}
\end{figure}


\begin{table}[t]
  \caption{Comparison of the time and area for our Matrix Multiplication module
           targeting Xilinx Artix 7 (\texttt{xc7a200t}) FPGA.}
  \label{tab:mat_mul}
  \centering
  {
    \fontsize{8pt}{10pt}\selectfont
    % \documentclass{standalone}
% 
% \usepackage{tabularx, multirow, booktabs, makecell}
% 
% \begin{document}

\newcommand{\mytimes}[1]{$#1, #1, #1$}

{%\setlength{\tabcolsep}{1.2ex} 
\sisetup{table-auto-round,group-minimum-digits=4,group-digits=integer}
\begin{widetable}{\textwidth}{@{}c
                                S[table-format=2]
                                S[table-format=4]
                                S[table-format=4]
                                S[table-format=2]
                                S[table-format=4]
                                S[table-format=1.1]
                                S[table-format=4]
                                S[table-format=3]
                                S[table-format=2.2]
                                %S[fixed-exponent=3, scientific-notation=fixed,]
                                S[fixed-exponent=3, scientific-notation=fixed, table-format=2.0e1,
                                   round-precision=1, round-mode=figures]
                             @{}}
   \toprule
    & & &\multicolumn{4}{c}{\textbf{Resources}} \\
   \cmidrule(lr){4-7}
   $d_1, d_2, d_3$ &  $p_{mm}$ & $q$ & \multicolumn{2}{c}{\textbf{Area}} & \multicolumn{2}{c}{\textbf{Memory}} & {\bf Cycles} & {\bf Freq.} & {\bf Time} & {\bf Time$\times$Area}\\
   \cline{4-5} 
   \cline{6-7}
   & & &{(LUT)} &{(DSP)} & {(FF)} & {(BRAM)} & {(cyc.)} & {(MHz)}  & {(us)} &  \\   

  \toprule
  \multicolumn{11}{c}{Parameter-set specific designs:}\\[1mm]
   \mytimes{14}  &  2   & 4093   & 279   & 2   & 197   & 0.5   & 1380   & 218   &  6.34   & 2000 \\
   \mytimes{22}  &  2   & 4093   & 294   & 2   & 206   & 0.5   & 5332   & 218   & 24.50   & 7000 \\
   \mytimes{30}  &  2   & 2039   & 298   & 2   & 206   & 0.5   &13508   & 195   & 69.20   & 21000 \\\midrule                                                                 
   \mytimes{14}  &  4   & 4093   & 383   & 4   & 317   & 1     &  792   & 218   &  3.64   & 1000 \\
   \mytimes{22}  &  4   & 4093   & 396   & 4   & 332   & 1     & 2912   & 218   & 13.38   & 5000 \\
   \mytimes{30}  &  4   & 2039   & 387   & 4   & 320   & 1     & 7208   & 218   & 33.11   & 13000 \\\midrule
   \mytimes{14}  &  8   & 4093   & 598   & 8   & 565   & 1.5   &  400   & 218   &  1.84   & 1000 \\
   \mytimes{22}  &  8   & 4093   & 625   & 8   & 580   & 1.5   & 1460   & 199   &  7.32   & 5000 \\
   \mytimes{30}  &  8   & 2039   & 606   & 8   & 556   & 1.5   & 3608   & 199   & 18.09   & 11000 \\
  \toprule
  \multicolumn{11}{c}{Joint designs:}\\[1mm]
   \mytimes{14}  &      &        &       &     &       &       & 1380   &       &  6.93   & 2000 \\
   \mytimes{22}  &  2   & both   & 335   & 2   & 220   & 0.5   & 5332   & 199   & 26.79   & 9000 \\
   \mytimes{30}  &      &        &       &     &       &       &13508   &       & 67.88   & 23000 \\
  \midrule                                                              
   \mytimes{14}  &      &        &       &     &       &       &  792   &       &  3.65   & 2000 \\
   \mytimes{22}  &  4   & both   & 460   & 4   & 348   & 1     & 2912   & 217   & 13.42   & 6000 \\
   \mytimes{30}  &      &        &       &     &       &       & 7208   &       & 33.23   & 15000 \\
  \midrule                                                              
   \mytimes{14}  &      &        &       &     &       &       &  400   &       &  2.04   & 2000 \\
   \mytimes{22}  &  8   & both   & 753   & 8   & 612   & 1.5   & 1460   & 196   &  7.43   & 6000 \\
   \mytimes{30}  &      &        &       &     &       &       & 3608   &       & 18.36   & 14000 \\
  \bottomrule
\end{widetable}}
% \end{document}


  }
\end{table}


Performance and resource demands
for the different parameter sets
and design variants are shown in \cref{tab:mat_mul}.
Since the arithmetic latencies
of the field-specific variants
and the joint design
are identical,
the number of cycles
depends only on the matrix dimensions
$d_1$, $d_2$, and $d_3$
as well as the degree of parallelization $\mmperf$.
The maximum frequency is in about the same range
for all cases
with about $10\%$ variance.
The joint design,
however,
requires more computational logic resources
for the joint field arithmetic
(see \cref{sec:finite_field_arithmetic})
and for supporting different matrix dimensions
in the same design.


\subsection{Matrix Systemization} \label{sec:systemizer}
\begin{table}[t]
  \centering
  \caption{Comparison of the time and area for our Systemizer module
           targeting Xilinx Artix 7 (\texttt{xc7a200t}) FPGA.}
  \label{tab:systemizer}
  {
    \fontsize{8pt}{10pt}\selectfont
    % \documentclass{standalone}
% 
% \usepackage{tabularx, multirow, booktabs, makecell}
% 
% \begin{document}

{%\setlength{\tabcolsep}{1.2ex} 
\sisetup{table-auto-round,group-minimum-digits=4,group-digits=integer}
\begin{widetable}{\textwidth}{@{}c
                                S[table-format=1]
                                S[table-format=4]
                                S[table-format=5]
                                S[table-format=2]
                                S[table-format=4]
                                S[table-format=2.1]
                                S[table-format=6]
                                S[table-format=3]
                                S[table-format=4.2]
                                % S[fixed-exponent=3, scientific-notation=fixed,]
                                S[fixed-exponent=3, scientific-notation=fixed, table-format=4.0e1,
                                   round-precision=3, round-mode=figures]
                             @{}}
   \toprule
    & & &\multicolumn{4}{c}{\textbf{Resources}} \\
   \cmidrule(lr){4-7}
    \textbf{Matrix Size} &  $p_{mm}$ & $q$ & \multicolumn{2}{c}{\textbf{Area}} & \multicolumn{2}{c}{\textbf{Memory}} & {\bf Cycles} & {\bf Freq.} & {\bf Time} & {\bf Time$\times$Area}\\
   \cline{4-5} 
   \cline{6-7}   rows$\times$cols & & &{(LUT)} &{(DSP)} & {(FF)} & {(BRAM)}    & {(cyc.)} & {(MHz)}  & {(us)} &  \\   
   \toprule

  \multicolumn{11}{c}{Parameter-set specific designs:}\\[1mm]
$14 \times 196$ & 2   & 4093   &  943   &  6   & 1076 & 1.5 &   9345 & 185     &   50.52   & 48000 \\
$22 \times 484$ & 2   & 4093   & 1067   &  6   & 1334 & 1.5 &  57393 & 177     &  324.21   & 346000 \\
$30 \times 900$ & 2   & 2039   & 1178   &  6   & 1511 & 1.0 & 199393 & 199     & 1002.15   & 1181000 \\\midrule
$14 \times 196$ & 4   & 4093   & 2172   & 20   & 2670 & 3.0 &   2718 & 177     &   15.39   & 33000 \\
$22 \times 484$ & 4   & 4093   & 2288   & 20   & 3155 & 3.0 &  15702 & 183     &   85.94   & 197000 \\
$30 \times 900$ & 4   & 2039   & 2388   & 20   & 3408 & 2.0 &  53222 & 173     &  308.32   & 736000 \\\midrule
$14 \times 196$ & 8   & 4093   & 6442   & 72   & 7030 & 7.5 &    794 & 193     &    4.12   & 27000 \\
$22 \times 484$ & 8   & 4093   & 6645   & 72   & 8192 & 7.5 &   4069 & 166     &   24.54   & 163000 \\
$30 \times 900$ & 8   & 2039   & 6783   & 72   & 8939 & 5.0 &  13490 & 170     &   79.24   & 537000 \\
  \toprule

  \multicolumn{11}{c}{Joint designs:}\\[1mm]
$14 \times 196$ &     &        &         &      &      &     &   9345 &         &   52.67   & 67000 \\
$22 \times 484$ & 2   & both   &  1263   &  6   & 1595 & 2.5 &  57393 & 177     &  323.47   & 409000 \\
$30 \times 900$ &     &        &         &      &      &     & 199393 &         & 1123.78   & 1419000 \\
  \midrule
$14 \times 196$ &     &        &         &      &      &     &   2718 &         &   15.72   & 44000 \\
$22 \times 484$ & 4   & both   &  2778   & 20   & 3685 & 5.0 &  15702 & 173     &   90.82   & 252000 \\
$30 \times 900$ &     &        &         &      &      &     &  53222 &         &  307.84   & 855000 \\
  \midrule
$14 \times 196$ &     &        &         &      &      &     &    794 &         &    4.69   & 39000 \\
$22 \times 484$ & 8   & both   &  8333   & 72   & 9138 & 10.0&   4069 & 169     &   24.04   & 200000 \\
$30 \times 900$ &     &        &         &      &      &     &  13490 &         &   79.71   & 664000 \\

  \bottomrule
\end{widetable}}

% \end{document}

  }
\end{table}

Our hardware design
for matrix systemization
follows the state-of-the art
such as
\cite{DBLP:journals/tc/ShoufanWMHK10,DBLP:conf/reconfig/WangSN16,CHES:WanSzeNie17,TCHES:CCDLNSW22,DBLP:conf/dac/ZhuZCZLWL23,TCHES:BCHKPSY23}:
We are using a processor array
of a quadratic shape
that processes column-blocks of the input matrix
in several rounds
consisting of several steps.
For a detailed description of the operation of the systemizer,
we refer to \cite[Section 3]{TCHES:CCDLNSW22}.

The design in \cite{TCHES:BCHKPSY23}
combines matrix systemization and matrix multiplication in one design
with the goal to share resources between these two operations.
However,
we are using a pipelined overall design
to increase signing and verification performance
and hence we need individual modules
for matrix systemization and multiplication.
Due to the pipelined overall design,
all modules are under load most of the time
(as shown in \cref{tab::config})
and resources are in use in parallel
--- and hence do not need to be shared to achieve efficiency.

The design in \cite{DBLP:conf/dac/ZhuZCZLWL23}
reduces memory cost
by streaming out parts of the systemized matrix
as complete columns become available during computation
without storing the entire matrix on-chip.
This optimization does not apply to MEDS
(except if external memory is used),
since the large $k \times nm$ matrices
$\mat{\tilde{G}_i}$ and $\mat{\hat{G}_i}$
need to be fed row-wise into the hash function.

In contrast to earlier designs
aiming at the systemization of matrices over $\F_2$
as they appear, e.g.,
in binary codes
used for the code-based McEliece cryptosystem
\cite{DBLP:journals/tc/ShoufanWMHK10,DBLP:conf/reconfig/WangSN16,TCHES:CCDLNSW22}
or matrices over binary fields
as used, e.g., in code-based \cite{CHES:WanSzeNie17}
as well as multivariate cryptosystems
\cite{ferozpuri2018high,TCHES:BCHKPSY23},
we are dealing with matrices over medium-sized prime fields,
which means that we are dealing with a longer arithmetic latency:
while binary field arithmetic in earlier work
has a latency of one cycle,
we decided to deal with the larger arithmetic latency
by using several clock cycles
for multiplication and addition.
This does not affect
the overall process of the systemization,
but we need to take the latencies into consideration
to avoid race conditions
during the parallel processing of that data.
For example,
our design needs to wait for different amounts of overhead cycles
between consecutive rounds
to wait for data of the previous round being written back to memory
depending on the number of column blocks
that are being processed in between.
Depending on the size of the processor array,
the arithmetic latencies add up
and we quickly get an overall latency
of the processor array
of dozens of cycles.
However,
in contrast to earlier designs
\cite{DBLP:conf/reconfig/WangSN16,CHES:WanSzeNie17,TCHES:CCDLNSW22}
we do pipeline computation
between consecutive rounds
as much as possible.

Similar to \cite{TCHES:BCHKPSY23},
we are not too concerned about systemization failures,
which happen only very rarely for our field sizes:
%
\label{sec::syst_failure_prob}
%
The probability that a
random $n \times m$ matrix ($m \geq n$)
over a finite field $\F_q$
has a systematic form
is the same as that of an $n \times n$ matrix over $\F_	q$ being full rank.
For large $n$ and $q$,
this probability is approximately
$1 - \frac{1}{q}$.
%
Hence, for $q = 4093$, we have a probability of success
for the \textsf{SF} operation
in \cref{alg::sign::SF}
of about $99.98\%$
and for $q = 2039$ of about $99.96\%$.
In other words,
we expect a failure to systemize a matrix over $\F_q$
roughly every $q$ attempts
---
for $q = 4093$ we expect one failure roughly every $4093$
and for $q = 2039$ roughly every $2039$ systemization attempts.
Therefore,
we omit some of the tweaks described in \cite{TCHES:CCDLNSW22}:
As we do not require a mechanism
to detect failure
{as early as possible
but only at the end of systemization,
we do not need store all operations
that are generated in pivoting steps
in-place in the data memory
for later use.
Instead,
we can use a relatively small additional memory
as in \cite{DBLP:conf/reconfig/WangSN16,CHES:WanSzeNie17,TCHES:BCHKPSY23}
to store the operations
only of the current pivoting step,
overwriting those of previous steps.
Similar to these designs,
we detect systemization failure
at the end of the computation
in the final pivoting step.

\begin{figure}[t]
  \centering
  \includegraphics[width=0.8\textwidth]{figs/systemizer_main.pdf}
  \caption{Systemizer module with $p_\text{sys}$ vector \glsplural{ALU}
               of width $p_\text{sys}$.
           }
       
  \label{fig::sys_overview}\label{fig::sys_row}
\end{figure}

\Cref{fig::sys_overview} shows an overview
of the systemizer module.
We use the performance parameter $p_\text{sys}$
to specify the number of both rows and columns
of the processor array.
The number of columns
corresponds to the column-block width
of the corresponding \gls{BRAM}
that stores the matrix data.
We are grouping the processor elements
of each row
into a vector \glspl{ALU} of width $p_\text{sys}$.

The design operates
by feeding matrix data
in column blocks
row-by-row into the processor array,
one row in each cycle.
Once a pivot matrix row reaches
the corresponding processor row,
it is stored in registers \texttt{SAR[i,j]}.
The pivoting rows need to be normalized,
which requires a scalar inversion
and a scalar-vector multiplication.
Later in the process,
incoming rows are reduced
by the pivoting row,
which requires a scalar multiplication
and a subtraction.

Now, as mentioned before,
the prime-filed arithmetic operations
have a considerable latency.
Since processor array
operates in a pipelined fashion,
the result of the normalization
is already required
for the next input.
Overcoming the latencies of the normalization
would require several buffer stages.
Also,
we would like to use only one vector unit
computing scalar-vector multiplication and vector addition,
not two for normalization and reduction.

Our solution to this problem
is to delay the normalization
of pivoting rows
until they are read out of the processor array.
This means that a new pivot row
can be stored in \texttt{SAR[i,j]} unmodified
and is available without further latency
in the next cycle.
For the normalization and reduction
we now can use just one single vector \gls{ALU},
but we need to prepare the respective inputs
at the cost of several cycles of latency.
The vector \gls{ALU} now always
multiplies \texttt{SAR[i,j]} by a scalar factor
either for normalization or for reduction
and adds an input vector to the result
which is either all zero for normalization
or input matrix data for reduction.
We need one additional scalar-scalar multiplication
per processor row
for normalizing the pivot row towards reducing follow-up rows.

The top part of \cref{fig::sys_row}
shows the pipelined modules
that prepare the inputs for
and control the behavior
of the vector unit
and the bottom part shows the vector unit itself
with $p_\text{sys}$ multiply-and-add units.
The overall latency
of each processor row
is defined by the sequential operation
of the top part
and the parallel latency of bottom part
as the latencies of
one multiplication,
one addition,
and one multiply-and-add operation
plus some additional latencies
to optimize data paths
of control logic.

The performance of our systemizer design
is shown in \cref{tab:systemizer}.
The maximum frequency ranges between
\SI{166}{\mega\hertz} and \SI{199}{\mega\hertz}.
The joint design requires the same number of cycles
but its maximum frequency is rather at the low end of the range.

\subsection{Pi Module}
\label{sec:pi}

The algorithm for the $\pi$ operation
is given in \cite[Algorithm 7]{NISTPQC-ADD-R1:MEDS23}
and was introduced in \cref{sec::notation}.
To recap,
the $\medspi{\mat{A}}{\mat{B}}{\mat{G}}$ operation
for $\mat{A} \in \Fq{m \times m}$,
$\mat{B} \in \Fq{n \times n}$, and
$\mat{G} \in \Fq{k \times mn}$
is defined as follows:
arrange the rows of matrix $\mat{G}$
into a $k$ separate $m \times n$ matrices
$\{\mat{P}_0, ... , \mat{P}_{k-1}\}$
and then compute $\mat{P}'_i = \mat{A} \cdot \mat{P}_i \cdot \mat{B}$.
Finally, we get the result as matrix $\mat{G}' \in \Fq{k \times mn}$
by packing $\mat{P}'_i$ into row $i$ of $\mat{G}'$.

In our implementation of the $\pi$ operation,
we avoid the first step
of arranging the rows of matrix $\mat{G}$
into $k$ separate $m \times n$ matrices.
Instead,
while generating matrix $\mat{G}$
using the \texttt{ExpandSystMat} module,
the matrix is already arranged
in $k$ separate BRAM units.
Therefore,
we can directly perform
the $\mat{A} \cdot \mat{P}_i \cdot \mat{B}$ operation
using our matrix multiplication unit
(described in \cref{sec:matrix_multiplication}).
We first perform $\mat{T} = \mat{A} \cdot \mat{P}$
and then $\mat{P}' = \mat{T} \cdot \mat{B}$.
\Cref{fig::pi_mod} shows the design of our $\pi$ module.

\begin{table}[t]
  \centering
  \caption{Comparison of the time and area for our Pi module
           targeting Xilinx Artix 7 (\texttt{xc7a200t}) FPGA
           for $p_{\pi mm} = 4$.}
  \label{tab:pi}
  {
    \fontsize{8pt}{10pt}\selectfont
    % \documentclass{standalone}
% 
% \usepackage{tabularx, multirow, booktabs, makecell}
% 
% \begin{document}

{%\setlength{\tabcolsep}{1.2ex} 
\newcommand{\mytimes}[1]{$#1, #1, #1$}
\sisetup{table-auto-round,group-minimum-digits=4,group-digits=integer}
\begin{widetable}{\textwidth}{@{}c
                                S[table-format=2]
                                S[table-format=4]
                                S[table-format=6]
                                S[table-format=3]
                                S[table-format=5]
                                S[table-format=2.1]
                                S[table-format=6]
                                S[table-format=3]
                                S[table-format=3.2]
                                %S[fixed-exponent=3, scientific-notation=fixed,]
                                S[fixed-exponent=3, scientific-notation=fixed, table-format=4.1e1,
                                   round-precision=1, round-mode=figures]
                             @{}}
   \toprule
    & & &\multicolumn{4}{c}{\textbf{Resources}} \\
   \cmidrule(lr){4-7}
   $k, m, n$ &  $p_{\pi pm}$ & $q$ & \multicolumn{2}{c}{\textbf{Area}} & \multicolumn{2}{c}{\textbf{Memory}} & {\bf Cycles} & {\bf Freq.} & {\bf Time} & {\bf Time$\times$Area}\\
   \cline{4-5} 
   \cline{6-7}
    & & &{(LUT)} &{(DSP)} & {(FF)} & {(BRAM)} & {(cyc.)} & {(MHz)}  & {(us)} &  \\   
   \toprule

  \multicolumn{11}{c}{Parameter-set specific designs:}\\[1mm]
\mytimes{14}  & 2      & 4093   & 1329   & 8      & 660    & 2      & 25474  & 218  &  117.03 & 155000  \\
\mytimes{22}  & 2      & 4093   & 1610   & 8      & 706    & 2      & 91354  & 218  &  419.68 & 675000  \\
\mytimes{30}  & 2      & 2039   & 1655   & 8      & 688    & 2      & 223474 & 218  & 1026.64 & 1697000  \\\midrule                                                                 
\mytimes{14}  & 4      & 4093   & 2172   & 16     & 1248   & 4      & 13588  & 218  &  62.42  & 135000  \\
\mytimes{22}  & 4      & 4093   & 2204   & 16     & 1313   & 4      & 48724  & 196  &  249.17 & 548000  \\
\mytimes{30}  & 4      & 2039   & 2457   & 16     & 1269   & 4      & 119188 & 196  &  609.53 & 1494000  \\\midrule                                                                 
\mytimes{14}  & 8      & 4093   & 3350   & 32     & 2438   & 8      & 6796   & 218  &  31.22  & 104000  \\
\mytimes{22}  & 8      & 4093   & 3526   & 32     & 2536   & 8      & 24364  & 218  &  111.93 & 394000  \\
\mytimes{30}  & 8      & 2039   & 4028   & 32     & 2437   & 8      & 59596  & 218  &  273.78 & 1101000  \\

  \toprule                                                              
  \multicolumn{11}{c}{Joint designs:}\\[1mm]                            

\mytimes{14}  &        &        &        &        &        &        & 25474  &      &  117.44 & 197000  \\
\mytimes{22}  & 2      & both   & 1891   & 8      & 751    & 2      & 91354  & 217  &  421.14 & 705000  \\
\mytimes{30}  &        &        &        &        &        &        & 223474 &      & 1030.22 & 1725000  \\
  \midrule                                                                 
\mytimes{14}  &        &        &        &        &        &        & 13588  &      &  62.64  & 190000  \\
\mytimes{22}  & 4      & both   & 2856   & 16     & 1392   & 4      & 48724  & 217  &  224.62 & 682000  \\
\mytimes{30}  &        &        &        &        &        &        & 119188 &      &  549.46 & 1668000  \\
  \midrule                                                                 
\mytimes{14}  &        &        &        &        &        &        & 6796   &      &  31.33  & 174000  \\
\mytimes{22}  & 8      & both   & 4732   & 32     & 2661   & 8      & 24364  & 217  &  112.32 & 623000  \\
\mytimes{30}  &        &        &        &        &        &        & 59596  &      &  274.74 & 1523000  \\
  \bottomrule



\end{widetable}}

% \end{document}

  }
\end{table}


We provide a performance parameter \pipmperf
for the $k$ matrix multiplications
such that we can select number
of matrix multiplications 
to be performed in parallel
(the ``pm'' strands for parallel matrix multiplications).
We control the number of parallel finite field operations
in the matrix multiplication units
of the $\pi$ operation individually
by setting the performance parameter \mmperf
of the matrix multiplication unit
(see \cref{sec:matrix_multiplication})
to \pimmperf for the $\pi$ operation.
This gives us two different performance parameters
for our $\pi$ design:
first, the number \pimmperf of prime field \glspl {ALU}
per matrix multiplication
and
second, the number \pipmperf
of parallel matrix multiplications.
%
The latency $l_\pi$ of our $\pi$ operation
is $l_\pi = \frac{k}{\pipmperf} \left( 2l_\text{mmpipe} + \frac{mmn + mnn}{\pimmperf} \right)$
%
with $l_\text{mm}$ and $l_\text{mmpipe}$
as defined in \cref{sec:matrix_multiplication}.

The synthesis results for our \texttt{Pi} module
for different configurations are
shown in \cref{tab:pi}.
The maximum frequency is similar for all variants.
The larger matrix dimensions for $k,m,m = 30$
seem to outweigh the smaller resource requirement
of $\F_{2039}$ arithmetic.
The resource overhead
of the joint design
compared to the field-specific variants
is moderate.


\subsection{Signing and Verification}\label{sec:signing}

\Cref{fig:sign} shows
the block diagram of the overall hardware design
combining both \gls{MEDS} sign and verify algorithms
detailed in \cref{alg:sign} and \cref{alg:verify} respectively by sharing hardware modules between both operations.

In sign, we assume that byte deserialization
of the input data for the secret key \sk
into the seed $\sigma_{\mat{G}_0}$
and the matrices $\mat{A}^{-1}_i$ and $\mat{B}^{-1}_i$
using the \texttt{Decompress} module
are performed on a higher layer.
We also leave the composition
of the signed message \signedmessage
including
the serialization of output data $\mu_i$ and $\nu_i$
using the \texttt{Compress} module
to a higher layer
and just provide read ports to its components
as output ports to the design.
We also omit the implementation
of the random number generator \textsf{Randombytes}
and instead assume that the random seed $\delta$
is provided from a higher level.

\begin{figure}[t]
  \centering
  \includegraphics[width=.975\textwidth]{figs/sign.pdf}
  \caption{High-level design overview of the \gls{MEDS} signing operation.}
  \label{fig:sign}
\end{figure}

Similarly, in verification, we assume that byte deserialization
of the input data for the public key \pk
into the seed $\sigma_{\mat{G}_0}$ and deserialization of signed message
$\signedmessage$ into $p$, $d$, $\alpha$, and $\mesg$
are performed at a higher layer.

We note that,
in our hardware design,
the operations
\texttt{XOF},
\texttt{ExpandInvMat},
\texttt{SeedTree}, and \texttt{PathToSeedTree}
share the same SHAKE256 module
as shown in \cref{fig:sign}
but that we use a separate SHAKE256 module that is shared between \texttt{ExpandSystMat}, \texttt{ParseHash}
for the combined \texttt{Compress} and \texttt{Hash} operation
(\cref{alg::sign::comm_hash} in \cref{alg:sign} and \cref{alg::verify::comm_hash} in \cref{alg:verify}).
The reason for this will be described
in the discussion of the loop pipeline
later on in this section.

The most expensive operations $\pi$ and \texttt{SF}
are in the main for-loops
of the sign and verify operations
i.e., \cref{alg::sign::comm_start} to \cref{alg::sign::comm_end}
in \cref{alg:sign} and \cref{alg::verify::comm_start} to \cref{alg::verify::comm_end}
in \cref{alg:verify} respectively.
Initially,
we had considered
to construct a large finite field vector \gls{ALU}
for the use in both $\pi$ and \texttt{SF}.
However,
the required multiplexing and control logic
likely would have been inefficient, complex,
error prone, and hard to maintain.
Also,
$\pi$ requires about two times more finite field multiplications
compared to \texttt{SF}
but no finite field inversions
and the different matrices
that are systemized
in \texttt{SF} and \texttt{ExpandInvMat}
have different dimensions.
A joint large finite field vector \gls{ALU}
would hence lead to a less fine-grained control
over performance vs.~area trade-offs
between those operations.
Therefore,
to obtain high performance and high efficiency,
we decided to implement this loop in a pipelined fashion
with four pipeline stages.

\medskip\noindent
\textbf{Stage 1:}\ 
%
This stage consists of operations
involved in \crefrange{alg::sign::startTildeABG}{alg::sign::gen_AB_end} of \cref{alg:sign} and \crefrange{alg::verify::munustart}{alg::verify::munuend} and  \crefrange{alg::verify::mucopy}{alg::verify::munuexit}  of \cref{alg:verify}. 
In signing, first a leaf seed ($\sigma_i$)
concatenated with the salt ($\alpha$)
and the iteration number ($i$)
is extended into two seeds 
$\sigma_{\tilde{A}_i}$, $\sigma_{\tilde{B}_i}$
using the \texttt{XOF} module. 
Then $\sigma_{\tilde{A}_i}$ and $\sigma_{\tilde{B}_i}$
are used to generate invertible matrices
$\tilde{A}_i$ and $\tilde{B}_i$. 
In our hardware design,
we accomplish the generation of
$\sigma_{\tilde{A}_i}$ and $\sigma_{\tilde{B}_i}$
using the \texttt{SHAKE256} module
and the \texttt{XOF} interface module
by selecting the ``XOF mode''.
The expanded seeds are stored in \texttt{sigmaAB} \gls{BRAM}.
Then,
the \texttt{ExpandInvMat} module
uses the seed $\sigma_{\tilde{A}_i}$
and the \texttt{XOF} interface
and \texttt{SHAKE256} module
by selecting the ``ExpandInvMat'' mode.
This generates data for matrix $\tilde{A}_i$
and the \texttt{Systemizer} module
inside the \texttt{ExpandInvMat} module is used to check
if the matrix is invertible or not.
If it is not invertible then,
the module draws another matrix
from the current \texttt{SHAKE256} state
and performs another check for invertability of the matrix.
This process is repeated
until an invertible matrix is found.
Once it is ensured that the matrix is invertible
then $\tilde{B}_i$ is generated in similar fashion.
We note that, for each iteration $i$, we store the related $\sigma_i$ value in a BRAM inside Stage 1 for later use 
in computing $\mu_i$ and $\nu_i$ matrices.

In verification, if the value of $h_i = 0$ then all operations in stage 1 are
similar to signing except here we use $\sigma_{\hat{A}_i}$ and
$\sigma_{\hat{B}_i}$ and generate invertible matrices $\mu_{i}$ and $\nu_{i}$.
And in the case of $h_i > 0$, we skip the sampling part inside the
\texttt{ExpandInvMat} and instead copy the $\mu_{i}$ and $\nu_{i}$ from the
input into the \gls{BRAM} where sampled $\mu_{i}$ and $\nu_{i}$ are stored and check
for their invertability. In case the $\mu_{i}$ or $\nu_{i}$ are non-invertible,
then a global ``systemization fail'' signal is generated, which will trigger ``invalid
signature'' output.

\medskip\noindent
\textbf{Stage 2:}\ 
%
This stage consists of the $\pi$ operation
(i.e., \cref{alg::sign::pi} of \cref{alg:sign} and \cref{alg::verify::pi} of \cref{alg:verify}). 
In signing, this operation involves
matrix multiplication of rows of $\mat{G}_0 \in \Fq{k \times mn}$
with matrix $\mat{\tilde{A}}_i \in \Fq{m\times m}$
and with $\mat{\tilde{B}}_i \in \Fq{n\times n}$.
As shown in \cref{fig:sign},
in our hardware design,
this operation is accomplished
by the \texttt{Pi} module using
$\mat{G}_0$ from {\tt G0 BRAM}
and $\mat{\tilde{A}}_i$, $\mat{\tilde{B}}_i$
(generated in Stage~1).
For verification,
similar operations are performed as for signing
except that $\mat{G}_{hi}$, $\mat{\mu}_{i}$, and $\mat{\nu}_{i}$
are being processed
in place of $\mat{G}_{0}$, $\mat{\tilde{A}}_{i}$, and $\mat{\tilde{B}}_{i}$.
 
\medskip\noindent
\textbf{Stage 3:}\ 
%
This stage consists of
operation \textsf{SF}
and the checking
if the $\mat{\tilde{G}}_i$
has systematic form
(i.e., \cref{alg::sign::SF}
of \cref{alg:sign} or \cref{alg::verify::SF} of \cref{alg:verify}).
We use a \texttt{Systemizer} module
(described in \cref{sec:systemizer})
to compute the systematic form.
As described in \cref{sec::syst_failure_prob},
there is a possibility
that the input matrix
does not have a systematic form.
We describe
how we handle this situation
in detail below.
This step is the same
for sign and verify
with corresponding inputs.
 
\medskip\noindent
\textbf{Stage 4:}\ 
%
This stage consists
of operations for compressing and hashing
matrices $\mat{\tilde{G}}_i$
together with the message \mesg
(i.e., \cref{alg::sign::comm_hash}
of \cref{alg:sign}).
Although the compress and hash operation
are not part of the for loop
in the algorithm,
in our hardware design
we move the compression
and the hash-absorb operation
inside the for loop.
This avoids the need
to store all $\mat{\tilde{G}}_i$ values
in the memory,
which would require a significant amount
of \gls{BRAM} storage on the FPGA.
For this,
we use a dedicated \texttt{SHAKE256} module
for Stage 4.
This \texttt{SHAKE256} module
digests each $\mat{\tilde{G}}_i$
and maintains the hash state.
Hence,
we only need to store
one $\mat{\tilde{G}}_i$
in the memory at a given time.
This step is the same
for sign and verify.
 

\medskip\noindent
After finishing all $t$ iterations of the loop,
the message input is absorbed into the state as well
and finally we squeeze the hash state
to compute the hash value~$d$.
 
\paragraph{Pipeline Control and Memory Buffering.}
%
The pipelined part of our hardware design
is highlighted in blue in \cref{fig:sign}.
The pipelining in our hardware design
varies from the traditional register-based pipelining:
Since the data we move
from one stage to another
are relatively large matrices,
we use \glspl{BRAM}
as buffers between the pipeline stages.
These memory buffers
are shown as blush pink blocks in \cref{fig:sign}.
The \texttt{MemCopy} modules
copy the data from the internal memories
of each stage
and move the data to the memory buffers,
which can then be consumed in the next stage.

As the workload of each stage is different,
the number of clock cycles
taken by each stage is also different.
Therefore,
to balance the clock cycles taken by each stage,
we introduce the following five performance parameters:


\begin{itemize}
  \item[\sw:]
     Stage~1 performance parameter
     that corresponds to the performance parameter $p_\text{sys}$
     (see \cref{sec:systemizer})
     of the matrix systemizer
     for checking the invertability
     of ${\tilde{A}_i}$ and ${\tilde{B}_i}$ in the sign operation
     respective $\mu_i$ and $\nu_i$
     in verify.
     \sw controls the number of rows and columns
     in the processor array used inside the matrix systemizer.

  \item[\sx:]
    Stage 2 performance parameter
    that corresponds to the performance parameter $\pimmperf$
    (see \cref{sec:pi})
    of the matrix multiplication unit
    inside the \texttt{Pi} module.
    \sx controls the vector width
    inside the matrix multiplication unit.  

  \item[\sm:]
    Stage 2 performance parameter
    that corresponds to the performance parameter $\pipmperf$
    of the \texttt{Pi} module
    used inside stage 2.
    \sm controls number of matrix multiplications
    performed in parallel.

  \item[\sy:]
    Stage 3 performance parameter
    that corresponds to the performance parameter $p_\text{sys}$
    (see \cref{sec:systemizer})
    of the large $k \times m \cdot n$ matrix systemizer
    used inside stage 3
    for computing the systematic form of ${\tilde{G}_i}$ in sign
    respective ${\hat{G}_i}$ in verify.
    \sy controls the number of rows and columns
    in the processor array used inside the matrix systemizer.

  \item[\sz:] 
    Stage 4 performance parameter
    that corresponds to the shifter width
    used inside the compress and hash module.
    Larger \sz leads to fewer memory accesses
    while compressing and hashing the matrices ${\tilde{G}_i}$ respective ${\hat{G}_i}$.
\end{itemize}


\paragraph{Choosing suitable performance parameters.} 
%
Recall that the goal behind choosing the performance parameters
is to balance the pipeline stages
so that all the modules in the pipeline
are busy with their respective workloads.
Let us look at the methodology we followed
using the example
(shown in \cref{tab::lat})
for the Security Level I parameter sets:
Firstly,
we fix the value of $\sm = 1$
and then we assign all possible values
to \sw, \sx, \sy, and \sz
based on the \gls{MEDS} parameters
up to $m=n=k=14$
as shown in \cref{tab::lat}.
To further lower the latency of stage 2,
the \sm parameter can be tweaked.
The latencies reported in \cref{tab::lat}
also include the clock cycles required
for buffering data from one stage to another.
Based on these latencies,
we chose a value for each \sw, \sx, \sy, and \sz
to meet our optimization goal
(performance or resource consumption),

As an example for parameters
with moderate resource requirement,
we can select
$\sw = 1, \sx = 7, \sy = 6$, and $\sz = 6$
(marked in \cref{tab::lat})
%3
such that the cycles counts are roughly equal.
Setting $\sm = 4$ in this example
brings down the cycle count of stage 2
to about $\num{12132} / 4 = \num{3033}$
(plus the unaffected overhead for data movement between the stages).

Based on this method,
we propose different configurations for specific parameter sets
as well as for a unified design
that aim at balancing the pipeline stages
in all parameter sets
as shown in \cref{tab::config}.
We propose two variants of configurations:
1) a balanced design choice and
2) a high performance design choice
as shown in \cref{tab::config}. The choice to parameters is not just limited to the combinations provided in \cref{tab::config}. Several such parameter combinations can be constructed based on resources available on the target device.
We note that when selecting the parameters for a unified design,  to avoid resource wastage, we limit the values of parameters based on $m$, $n$, and $k$ values of smallest parameter set. 

\begin{table}[t]
  \centering
  \caption{Latencies of all pipeline stages for the Security Level 1 parameter sets
           with performance parameter $\sm = 1$.}
  \label{tab::lat}
  {
     \fontsize{8pt}{9pt}\selectfont
    


% \documentclass{standalone}
% 
% \usepackage{tabularx, multirow, booktabs, makecell}
% 
% \begin{document}

{%\setlength{\tabcolsep}{1.2ex} 
\sisetup{table-auto-round,group-minimum-digits=4,group-digits=integer}
\begin{widetable}{\textwidth}{@{}S[table-format=1]
                                S[table-format=4]
                                S[table-format=5]
                                S[table-format=5]
                                S[table-format=5]
                                S[table-format=5]                              
                             @{}}
   \toprule
   {$\sw = \sx = \sy  = \sz$} & {Stage 1} & {Stage 2} & {Stage 3} & {Stage 4}   \\   
   \toprule
  %  \midrule
    1 & \cellcolor{black!10} 3534    & 83364   & 48191   & 16599   \\
    2 & 1432    & 41812   & 14877   & 8521    \\
    3 & 1142    & 29926   & 8350    & 5883    \\
    4 & 1104    & 23990   & 5702    & 4503    \\
    5 & 1042    & 18054   & 4020    & 3751    \\
    6 & 1120    & 18054   & \cellcolor{black!10} 3557    & \cellcolor{black!10} 3178    \\
    7 & 980     & \cellcolor{black!10} 12132   & 2477    & 2782    \\
    8 & 1032    & 12118   & 2294    & 2522    \\
    9 & 1084    & 12118   & 2153    & 2484    \\
   10 & 1136    & 12118   & 2054    & 2371    \\
   11 & 1188    & 12118   & 1955    & 2652    \\
   12 & 1240    & 12118   & 1912    & 2197    \\
   13 & 1292    & 12118   & 1869    & 2137    \\
   14 & 975     & 6195    & 1643    & 1796   \\

  \bottomrule
\end{widetable}}

% \end{document}

  }
\end{table}

\begin{table}[t]
  \centering
  \caption{Configuration table showing selected configurations for $\sw, \sx,
           \sy, \sz, \sm$ and idle percentages for the different stages for each
           configuration. In case of the unified design the idle percentages are
           average of different security levels.}
  \label{tab::config}
  {
     \fontsize{8pt}{10pt}\selectfont
     


% \documentclass{standalone}
% 
% \usepackage{tabularx, multirow, booktabs, makecell}
% 
% \begin{document}

  \centering
  \setlength{\tabcolsep}{0.6ex} 
  % \sisetup{table-auto-round,group-minimum-digits=4,group-digits=integer}
 \begin{tabularx}{\textwidth}{@{}
 				         l
                 l
                 c
                 S[table-format=1]
                 S[table-format=2]
                 S[table-format=2]
                 S[table-format=2]
                 S[table-format=1]
                 c
                 c
                 c
                 c
                 @{}} 
 \toprule
  \multirow{2}{*}{\bf Design Choice} & \multirow{2}{*}{\textbf{Parameter Set}} &  \multirow{2}{*}{$q$} & \multicolumn{5}{c}{\textbf{Perf. Metrics}} & \multicolumn{4}{c}{{\textbf{Stage-wise Idle \%}}}  \\%& \sw & \sx & \sy & \sz & \sm\\
 \cmidrule(lr){4-8} \cmidrule(lr){9-12}
  & & & \sw & \sx & \sy & \sz & \sm & {\bf 1} & {\bf 2} & {\bf 3} & {\bf 4} \\
  \toprule
\multirow{4}{*}{\bf Balanced} & Level I                 & 4093   & 1 & 7  & 6  & 6  & 3 & {3.6\%} & {0.0\%} & {3.4\% } & {6.1\% } \\ 
                              &  Level III              & 4093   & 1 & 11 & 8  & 9  & 5 & {0.5\%} & {1.7\%} & {0.0\% } & {0.5\% } \\ 
                              & Level V                 & 2039   & 1 & 10 & 10 & 12 & 5 & {0.0\%} & {4.4\%} & {2.8\% } & {5.8\% } \\ [2pt]
                              & {Unified Level I--V}    & {both} & 1 & 10 & 10 & 12 & 5 & {3.4\%} & {5.5\%} & {15.4\%} & {15.7\%} \\
 \midrule
\multirow{4}{*}{\bf High Perf.} &  Level I              & 4093   & 2 & 14 & 14 & 14 & 4 & {5.7\% } & {3.8\%} & {2.4\% } & {0.0\% } \\
                                & Level III             & 4093   & 2 & 11 & 22 & 22 & 8 & {1.5\% } & {0.3\%} & {0.0\% } & {3.9\% } \\
                                & Level V               & 2039   & 2 & 15 & 14 & 15 & 8 & {0.2\% } & {2.7\%} & {0.6\% } & {0.0\% } \\[2pt]
                                & {Unified Level I--V}  & {both} & 2 & 14 & 14 & 14 & 4 & {18.4\%} & {1.3\%} & {17.5\%} & {13.9\%} \\
  
 \bottomrule
 %\hline
 %\hline
 \end{tabularx}
% \end{tabular}}

% \end{document}

  }
\end{table}
 

\paragraph{Data flow.}
%
To cope with the different ready-times
of each stage
that results from the different cycle counts,
we resort to a handshake-based control mechanism
to control the flow of data between the stages.
After each stage is done with its computation,
it sends a handshake signal to the next stage
to indicate that the output data is ready for consumption
and waits for the response from the next stage.
This handshake mechanism
is handled by the \texttt{PipelineController} module
shown in \cref{fig:sign}.

As mentioned above
in Stage 3,
an additional challenge
in the pipeline design
is the handling of possible systemization failures
in some iteration $i$ in Stage 3.
In this case,
we need to repeat the operations
of the previous stages
starting again at iteration $i$.
We flush all the data from Stages 1 and 2,
which are working on data
that belongs to iterations $i+1$ and $i+2$,
restart Stage 1 from iteration $i$,
and step by step refill the pipeline.
Stage 4 is not affected by this
as it is working on data that belongs to iteration $i-1$.
The reason we restart from iteration $i$ is
because of the on-the-fly \texttt{CompressHash} operation
in Stage 4.
The data fed into the \texttt{SHAKE256} module
has to be in sequential order
to produce the correct hash value.
This means that we would need memory buffers
after Stage 3
to store data related to iterations $i+1$ and $i+2$
and wait until data from iteration $i$ becomes ready
to be loaded into module \texttt{CompressHash}.
We avoid this expensive memory buffering
by simply restarting the pipeline. 
As the failure probability of the systemizer is low
(as specified in \cref{sec::syst_failure_prob}),
the overhead of restarting the pipeline is marginal.
To restart Stage 1 with iteration $i$,
Stage 1 requires to backup the seed for $i$,
i.e., $\sigma_{i}$ generated in \cref{alg::sign::seed_xof}
of \cref{alg:sign}.
After Stage 3 has completed successfully,
the stored $\sigma_{i}$ is discarded.
The logic related to flushing and restaring
is also handled by the \texttt{PipelineController} module. 

\paragraph{Other operations.}
%
After the loop is finished,
the \texttt{CompressHash} module
requests the message \mesg as data stream 
from input port \texttt{message\_in}
and the message is loaded
into the \texttt{SHAKE256} module
to compute the hash value $d$.
Following that, for verification, this $d$ value is compared against the input $d$ 
value to ensure the verfication of the signature. Where as for signing,
the value $d$ is then loaded
into the \texttt{ParseHash} module
shown in \cref{fig:sign},
which parses the hash value
and generates the vector $h$.
The \texttt{\tt ParseHash} module
also captures $i$ values where $h_i > 0$
while generating $t$.
This is useful
for the following computation
of $\mu_i$ and $\nu_i$.
Rather than iterating over all $t$ values,
we only iterate over the $w$ indices where $h_i > 0$.
To compute matrices $\mu_i$ and $\nu_i$,
  We first generate the matrices $\mat{\tilde{A}}_i$ and $\mat{\tilde{B}}_i$ using 
  the $\sigma_i$ seeds stored in Stage 1 and by reusing the logic from Stage 1. 
  The reason for regenerating the matrices $\mat{\tilde{A}}_i$ and $\mat{\tilde{B}}_i$ 
  is to avoid storing the $t$ - $n \times n$ matrices and $t$ - $m \times m$ matrices, 
  which would otherwise consume a significant amount of BRAM. Therefore, we opted for 
  regenerating the matrices instead.
Then,
we reuse the \texttt{MatMul} units
of the \texttt{Pi} module.
The result is then stored
in \glspl{BRAM} \texttt{mu} and \texttt{nu} respectively.
If the \texttt{Pi} module
is configured to use
multiple matrix multiplication units,
the \texttt{MuNuComputation} module
computes $\mu_i$ and $\nu_i$ matrices in parallel.
In parallel to the computation of $\mu_i$ and $\nu_i$,
the $h$ vector along with $\rho$ and $\alpha$
are loaded into the \texttt{SeedTreeToPath} module,
which generates the seed path.
The $\mu_i$, $\nu_i$, $p$, $d$, $\alpha$ values
finally can be accessed
by a top-level module
through the output ports {\tt mu}, {\tt nu}, {\tt path}, {\tt d}, and {\tt alpha} respectively
as shown in \cref{fig:sign}.


\section{Evaluation}
\label{sec::eval}


\begin{table}[p]
\centering
\begin{adjustwidth}{-1cm}{}
{
  \fontsize{8pt}{10pt}\selectfont
  


% \documentclass{standalone}
% 
% \usepackage{tabularx, multirow, booktabs, makecell}
% 
% \begin{document}
%
{\setlength{\tabcolsep}{1ex} 
\sisetup{table-auto-round,group-minimum-digits=4,group-digits=integer}
\begin{tabularx}{\linewidth}{@{}
                                X
                                S[table-format=2.1]
                                S[table-format=3]
                                S[table-format=2.1]
                                S[table-format=3]
                                S[table-format=4]
                                S[table-format=4.1]
                                S[table-format=3]
                                S[table-format=4.1]
                                S[table-format=3]
                                S[table-format=3]
                                S[table-format=2.1]
                                S[table-format=2.1]
                             @{}
}

  \toprule
   \multirow[c]{3.8}{*}{\makecell{\textbf{MEDS}\\\textbf{Parameter Set}}}
   %\textbf{MEDS} 
      & \multicolumn{4}{c}{\textbf{Resources}} & \multicolumn{5}{c}{\textbf{Time}} & \multicolumn{3}{c}{\textbf{Improvment}} \\
   \cmidrule(lr){2-5} \cmidrule(lr){6-10} \cmidrule(lr){11-13}
   %\textbf{Param.} 
      & \multicolumn{2}{c}{\textbf{Area}} &
\multicolumn{2}{c}{\textbf{Memory}}  & {\bf Freq.} & \multicolumn{2}{c}{\textbf{Sign}} & \multicolumn{2}{c}{\textbf{Verify}} & {\bf cyc} & {\bf t} & {\textbf{T$\times$A}} \\
   \cmidrule(lr){2-3} 
   \cmidrule(lr){4-5}
   \cmidrule(lr){7-8}
   \cmidrule(lr){9-10}
  %  \cmidrule(lr){11-12}
   %\textbf{Set} 
      & {LUT} &{DSP} & {FF} & {BRAM} & {MHz} & {Mcyc} & {ms}& {Mcyc} & {ms} &  &  &  \\
      & {$\times 10^3$} &{} & {$\times 10^3$} & {} & {} & {} & {}& {} &  &  &  & {$\times 10^6$} \\
   \toprule
  % \multicolumn{11}{c}{\textbf{ Specific Design}} \\ \midrule
%    \multicolumn{13}{c}{\textbf{ Security Level I }}\\ \midrule
    \multicolumn{13}{c}{\textbf{Balanced -- \{\sw, \sx, \sy, \sz, \sm\} = \{1, 7, 6, 6, 3\}}}\\
			
			
MEDS-9923     & 24.3 & 66  & 20.5 & 111.5   & 116 & 7.5 & 65 & 7.4 & 64 & 69 & 4.1 & 3.1\\
MEDS-13220    & 21.6 & 66  & 19.5 & 69.5    & 130 & 1.3 & 10 & 1.2 & 10 & 68 & 4.6 & 0.4 \\
\midrule
\multicolumn{13}{c}{\textbf{High Performance -- \{\sw, \sx, \sy, \sz, \sm \} = \{2, 14, 14, 14, 3\}}}\\
MEDS-9923  & 46.8 & 259 & 44.8 & 192.5   & 115 & 3.3 & 29 & 3.3 & 29 & 154 & 9.3  & 2.7 \\
MEDS-13220 & 44.1 & 259 & 43.8 & 150.5   & 132 & 0.6 & 4  & 0.6 & 4  & 154 & 10.6 & 0.4 \\
\toprule
%\multicolumn{13}{c}{\textbf{ Security Level III}}\\    \midrule
\multicolumn{13}{c}{\textbf{Balanced -- \{\sw, \sx, \sy, \sz, \sm \} = \{1, 11, 8, 9, 5\}}} \\
MEDS-41711    & 34.0 & 130 & 31.2 & 205.5& 123 & 9.8 & 80 & 9.5 & 78 & 151 & 9.8  & 5.3 \\
MEDS-69497    & 33.1 & 130 & 30.7 & 187	 & 125 & 2.9 & 23 & 2.5 & 20 & 141 & 9.2 & 1.4 \\
\midrule
\multicolumn{13}{c}{\textbf{High Performance -- \{\sw, \sx, \sy, \sz, \sm \} = \{2, 11, 22, 22, 8\}}}\\
MEDS-41711 & 80.5 & 601 & 91.0 & 204.5	 & 129 & 6.1 & 47 & 5.9 & 46 & 242 & 16.4 & 7.6\\ 
MEDS-69497 & 79.6 & 601 & 90.5 & 186	   & 129 & 1.8 & 14 & 1.5 & 12 & 227 & 15.4 & 2.1\\
\toprule
%\multicolumn{13}{c}{\textbf{ Security Level V }}\\  \midrule
\multicolumn{13}{c}{\textbf{Balanced -- \{\sw, \sx, \sy, \sz, \sm \} = \{1, 10, 10, 12, 5\}}}\\
MEDS-134180   & 38.3 & 163 & 39.5 & 284.5   & 130 & 11.2 & 87 & 9.0 & 70 & 160 & 10.9 & 5.9 \\
MEDS-167717   & 38.0 & 163 & 39.4 & 327.5   & 133 & 8.1  & 61 & 5.3 & 40 & 141 & 9.8  & 3.9 \\
\midrule
\multicolumn{13}{c}{\textbf{High Performance -- \{\sw, \sx, \sy, \sz, \sm \} = \{2, 15, 14, 15, 8\}}}\\
MEDS-134180 & 61.5 & 337 & 60.9 & 290   & 127 & 5.4 & 43 & 4.5 & 36 & 322 & 21.5 & 4.9 \\
MEDS-167717 & 60.6 & 337 & 60.9 & 288   & 131 & 3.9 & 29 & 2.7 & 21 & 290 & 19.9 & 3.0 \\
\toprule
\multicolumn{13}{c}{\textbf{ Unified Design}} \\ \midrule
\multicolumn{13}{c}{\textbf{Balanced -- \{\sw, \sx, \sy, \sz, \sm\} = \{1, 10, 10, 12, 5\}}}\\
%All & 44930 & 164 & 42485 & 340   & 113 &      \\
MEDS-9923   & {\multirow{6}{*}{\num{44.8}}} 
            & {\multirow{6}{*}{\num{163}}} 
            & {\multirow{6}{*}{\num{43.1}}} 
            & {\multirow{6}{*}{\num{344.5}}} 
            & {\multirow{6}{*}{\num{113}}} 
                          & 5.2  & 46  & 5.1  & 46  & 100 & 5.9  &  4.1\\
MEDS-13220  &  &  &  &  & & 0.9  & 8   & 0.8  & 8   & 98  & 5.8       &  0.7\\
MEDS-41711  &  &  &  &  & & 13.4 & 119 & 13.1 & 116 & 110 & 6.5       &  10.6\\
MEDS-69497  &  &  &  &  & & 3.8  & 34  & 3.4  & 31  & 105 & 6.2       &  2.9\\
MEDS-134180 &  &  &  &  & & 11.2 & 100 & 9.0  & 80  & 160 & 9.4       &  8.0\\
MEDS-167717 &  &  &  &  & & 8.1  & 72  & 5.3  & 47  & 141 & 8.3       &  5.3\\

\midrule
\multicolumn{13}{c}{\textbf{High Performance -- \{\sw, \sx, \sy, \sz, \sm \} = \{2, 14, 14, 14, 4\}}}\\
MEDS-9923   & {\multirow{6}{*}{\num{72.8}}} 
            & {\multirow{6}{*}{\num{273}}} 
            & {\multirow{6}{*}{\num{63.4}}} 
            & {\multirow{6}{*}{\num{386.5}}} 
            & {\multirow{6}{*}{\num{113}}} 
                          & 3.3  & 33  & 3.3  & 33  & 154 & 8.2 & 4.2 \\
MEDS-13220  &  &  &  &  & & 0.5   & 6  & 0.5  & 5   & 154 & 8.2      & 0.7 \\
MEDS-41711  &  &  &  &  & & 10.2 & 100 & 10.1 & 99  & 144 & 7.7      & 12.9 \\
MEDS-69497  &  &  &  &  & & 2.7  & 27  & 2.6  & 25  & 145 & 7.7      & 3.3 \\
MEDS-134180 &  &  &  &  & & 11.1 & 109 & 10.7 & 104 & 149 & 7.9      & 13.7 \\
MEDS-167717 &  &  &  &  & & 6.7  & 67  & 6.2  & 61  & 146 & 7.8      & 8.2 \\
\toprule
   \multicolumn{13}{c}{\textbf{Reference Software Implementation on an AMD Ryzen 7 PRO 5850U CPU} \cite{NISTPQC-ADD-R1:MEDS23}} \\
   %\multicolumn{13}{c}{\textbf{Reference Software Implementation}} \\
   %\multicolumn{13}{c}{\textbf{on an AMD Ryzen 7 PRO 5850U CPU} \cite{NISTPQC-ADD-R1:MEDS23}} \\
\midrule
MEDS-9923   & {---} & {---} & {---} & {---} & 1900 &  518.0 &  273 &  515.5 &  271 & {---} & {---} & {---} \\ 
MEDS-13220  & {---} & {---} & {---} & {---} & 1900 &   88.9 &   47 &   87.4 &   46 & {---} & {---} & {---} \\ 
MEDS-41711  & {---} & {---} & {---} & {---} & 1900 & 1467.0 & 772  & 1461.9 & 769  & {---} & {---} & {---} \\ 
MEDS-55604  & {---} & {---} & {---} & {---} & 1900 &  387.2 &  204 &  380.7 &  200 & {---} & {---} & {---} \\ 
MEDS-134180 & {---} & {---} & {---} & {---} & 1900 & 1629.8 & 858  & 1612.5 & 849  & {---} & {---} & {---} \\ 
MEDS-167717 & {---} & {---} & {---} & {---} & 1900 &  961.8 &  506 &  938.8 &  494 & {---} & {---} & {---} \\ 
   
  \bottomrule

\end{tabularx}
}

% \end{document}

}

  \caption{%
    Performance data
    for our \texttt{Sign} and \texttt{Verify} modules
    targeting a Xilinx Artix 7 (\texttt{xc7a200t}) FPGA. 
    Improvements are calculated as
    $T \times A = (t_{sign} + t_{verify})\times LUT$
    $cyc = \frac{(Sign + Verify\ Cycles\ for\ Software)}{(Sign + Verify\ Cycles\ for\ Hardware)}$,
    $t = \frac{(Sign + Verify\ time\ for\ Software)}{(Sign + Verify\ time\ for\ Hardware)}$.
  }
  \label{tab:sign}
\end{adjustwidth}
\end{table}


\Cref{tab:sign} shows the time and area results
for our MEDS signing and verification hardware
design for all parameter sets
targeting an AMD {\tt xc7a200t-3} FPGA.
We note that the maximum clock frequency
is limited by two different factors:
1)~In case of MEDS-9923,
the critical path lies inside the duplicate detection logic
inside the {\tt ParseHash} module.
We use a bit-vector mapping technique
to perform the duplicate detection
and since the value $t$ is large in case of MEDS-9923,
the fully combinatorial variable shifter and comparator is quite large.
2)~In all other cases,
the critical path lies inside the sampling unit
of \texttt{XOF} interface.

As discussed in \cref{sec:signing},
we propose two different design variants
for our implementation:
``Balanced'' and ``High Performance''.
The results for these choices
are presented in \cref{tab:sign}.
While the high performance design
requires only \SIrange{4.5}{29.5}{\milli\second}
for signing and a similar amount of time for verification,
it also takes up significant resources,
mainly in terms of \gls{BRAM} utilization.
We note that while the storage
required for the data
remains the same in both the design choices,
due to the wider memory word width
in the High Performance variant,
the synthesis tool needs to use multiple \gls{BRAM} units.

Additionally,
for each of the design variants,
the design allows further flexibility
to be synthesized for a specific parameter set
or for a unified design that allows us
to choose any of the six different parameter choices at run-time.
While the unified design
does not impact the cycles taken for signing and verification
compared to specific designs,
it does have an impact on resource utilization.
However,
it can be seen from \cref{tab:sign}
that the resource utilization of the unified design
is close to the utilization
of the parameter set specific designs
for the biggest parameter sets
(i.e., MEDS-134180 and MEDS-167717). 
 
We note that to the best of our knowledge
this is the first and only MEDS hardware implementation.
Therefore,
our primary comparison
is with the optimized software implementation
provided along with the MEDS specification \cite{NISTPQC-ADD-R1:MEDS23}.
We note that
while our design is running at a frequency range
of \SIrange{115}{133}{\mega\hertz},
the software implementations results are reported
for a CPU running at a frequency of \SI{1.9}{\giga\hertz}.
We note that both of our design choices
outperform the optimized software implementation
by a significant margin
as shown in \cref{tab:sign}. 
Our hardware implementation 
achieves a speed-up of $69\times$ to $322\times$ in sign + verify cycles compared to the software implementation. 
Similarly, it achieves a speed-up of $4.1\times$ to $21.5\times$ in sign + verify time, despite operating at a 
significantly lower frequency than the processor used in the software implementation. This answers \cref{RQ:1} positively, showing that MEDS algorithms provide inherent parallelism which 
can be exploited through hardware implementation to provide substantial speed-up as demonstrated by our results (in \cref{tab:sign}).

\paragraph{Comparison to Related Work.}
%
\Cref{tab_related_work_compare} presents a comparison to a number of other
\gls{PQC} signature schemes that have been
presented in literature.
We are unaware of other MEDS hardware implementations,
so no other MEDS works are included in the table.
Many works present data for the same XC7A100T FPGA that we use
and they generally report similar frequencies as we.
Specific comparisons with these schemes to our implementation
is difficult as different schemes
are based on different mathematical problems.
Further,
some related works use high-end FPGAs,
which generally give much better performance.


\afterpage{%
\begin{landscape}
\begin{table}[p]
\centering
\footnotesize

\sisetup{table-auto-round,group-minimum-digits=4,group-digits=integer}
\setlength{\tabcolsep}{0.6ex} 
\begin{widetable}{\textwidth}{
             @{}
             c
             c
             c
             S[table-format=3]
             S[table-format=5.2]
             S[table-format=3.2]
             c
             c
                             @{}
}
\toprule

\textbf{Design} & \textbf{Algorithm}                 & \textbf{FPGA} & \textbf{Freq.} & \textbf{Sign} & \textbf{Verify} & \textbf{Resources} & \textbf{T$\times$A} \\
                &                                    &               & {(MHz)} &  {(ms)}              &  {(ms)} & {(LUT/FF/DSP/BR)} & \\%$\frac{(t_s + t_v)\times LUT}{10^6}$\\[0.4ex]
% \hline

\toprule

 \cite{TCHES:BCHKPSY23} & ov-Ip-pkc                           & XC7A200T & 100 & 0.08 & 0.69         &  37k/25k/2/81                            &\phantom{1,,}0.03\\
 \cite{TCHES:BCHKPSY23} & ov-V-pkc+skc                        & XC7A200T & 100 & 28.57 & 3.93        &  83k/41k/4/359                           &\phantom{1,,}2.70\\
\midrule                    

 \cite{DBLP:conf/cosade/SayariMAKS24} & MAYO 1                 & Arm/Zynq-7020 & 100   & 28.6 & {---}   & 21k/13k/11/129                        &\phantom{1,,}0.60\\
 \cite{cryptoeprint:2023/1267} & MAYO 1                        & Artix-7 & 75   & 0.43 & 0.05      & 106k/38k/2/45.5                            &\phantom{1,,}0.05\\
\midrule                    

 \cite{DBLP:journals/tches/DeshpandeHSY24} & SDitH L1 GF256   & XC7A200T & 164   & 41.0 & 52.9     & 17k/9k/0/164.5                             &\phantom{1,,}1.60\\
 \cite{DBLP:journals/tches/DeshpandeHSY24} & SDitH L3 GF251   & XC7A200T & 164   & 276.1 & 183.6   & 34k/31k/472/521.5                          &\phantom{1,}15.63\\
\midrule

 \cite{SP:dPRS23} & Raccoon-128  2 shares                  & RISC-V/XC7A100T & 78   & 30.7 & 18.4        & \multirow{ 2}{*}{10k/4k/3/---}       &\phantom{1,,}0.49\\
 \cite{SP:dPRS23} & Raccoon-128 32 shares                  & RISC-V/XC7A100T & 78   & 284.10 & 17.86     &                                      &\phantom{1,,}3.02\\
\midrule 

 \cite{SAC:WJWDGSN19} & XMSS SHA256 $h=10$            & RISC-V/Cyclone V & 145   & 9.95 & 5.80      & 7k/10k/---/145                            &\phantom{1,,}0.11\\
\midrule 

 \cite{DBLP:conf/cardis/LandSG21} & Dilithium-III F           & XC7A100T & 145   & 0.85 & 0.23       & 30k/11k/45/21                            &\phantom{1,,}0.03\\
 \cite{DBLP:conf/fpt/BeckwithNG21} & Dilithium-V              & Artix-7 & 116   & 0.21 & 0.12        & 53k/28k/16/29                            &\phantom{1,,}0.02\\
\midrule 

 \cite{DBLP:journals/iacr/Saarinen24} & SLH-DSA-SHAKE-128f & RISC-V/XC7A100T & 100   & 49.0 & 4.4        & \multirow{ 2}{*}{14k/---/---/---}    &\phantom{1,,}0.75\\
 \cite{DBLP:journals/iacr/Saarinen24} & SLH-DSA-SHA2-256s & RISC-V/XC7A100T & 100   & 69620.1 & 8.9      &                                      &974.81\\
\midrule

 \cite{DBLP:conf/dsd/AmietLCZ20} & SPHINCS+-128f-simple       & XC7A100T & 250  & 1.01 & 0.16        & 48k/73k/1/11.5                           &\phantom{1,,}0.06\\
 \cite{DBLP:conf/dsd/AmietLCZ20} & SPHINCS+-256s-robust       & XC7A100T & 250  & 36.1 & 0.20        & 50k/76k/1/30                             &\phantom{1,,}1.82\\
\midrule

 \cite{DBLP:journals/iacr/SchmidAWZW23} & Falcon-512          & ZCU104 & 187.5  & 4.2 & {---}       & 23k/26k/101/23                            &\multirow{2}{*}{\phantom{1,,}0.17}\\
 \cite{DBLP:journals/iacr/SchmidAWZW23} & Falcon-512          & ZCU104 & 214.3  & {---} & 0.62      & 12k/8k/15/13                              &\\
 \cite{DBLP:journals/iacr/SchmidAWZW23} & Falcon-1024         & ZCU104 & 187.5  & 8.7 & {---}       & 45k/41k/182/37                            &\multirow{2}{*}{\phantom{1,,}0.58}\\
 \cite{DBLP:journals/iacr/SchmidAWZW23} & Falcon-1024         & ZCU104 & 214.3  &{---} & 1.3        & 13k/9k/2/4                                &\\

\bottomrule

\end{widetable}



\caption{Existing FPGA-based hardware implementations of various PQC signature
         schemes. For the listed related work, if the prior work implemented
         different variants of an algorithm, the fastest design is listed. The
         ``---'' indicates that the corresponding parameter was not specified or
         not implemented. {\bf T$\times$A} = $\frac{(t_{sign} + t_{verify})\times LUT}{10^6}$}
\label{tab_related_work_compare}
\end{table}
\end{landscape}
}

Compared to the hardware (co-)designs
of some other \gls{PQC} schemes
shown in \cref{tab_related_work_compare},
our hardware implementation of \gls{MEDS} 
(results shown in \cref{tab:sign}) makes MEDS a competitive choice when the time area product ({\bf T$\times$A}) is considered.
Some of the other designs are much more efficient
than our implementation,
however
this is due to the high computational cost
inherent to the \gls{MEDS} specification
in particular compared to lattice-based schemes.
Nevertheless,
the performance of our implementation
is comparable to several other \gls{PQC} schemes
such as SPHINCS+/SLH-DSA, SDitH, and Raccoon.
We believe that this shows
that the quality of our implementation
is on par with other work in this field.

Our results emphasize that \gls{MEDS} is computationally expensive
but that \gls{FS}-schemes with a large round number
can be parallelized efficiently
and can achieve high performance (at high resoruce cost).
Possible future works include
the implementation of key generation
and of recent proposals
for the optimization of \gls{MEDS}
as well as the exploration
of supporting different fields
in one joint design
for other applicable schemes
such as
CROSS \cite{NISTPQC-ADD-R2:CROSS24},
LESS \cite{NISTPQC-ADD-R2:LESS24},
and PERK \cite{NISTPQC-ADD-R2:PERK24}.


\section{Conclusion}
\label{sec::concl}

In the introduction in \cref{sec:intro},
we raised three research questions:

\medskip\noindent
\Cref{RQ:1} asks
if there is sufficient inherent parallelism in \gls{MEDS}
to speed up the sign and verify operations.
Given that there is plenty parallelism on the low level
multiplying and systemizing matrices
as well as on the high level
iterating over $t$ independent computations,
there is indeed ample opportunity to accelerate \gls{MEDS}.
We provide performance parameters
to control the low level parallelization
and we pipeline the main loop,
achieving a significant speed up
for \gls{MEDS} signing and verification.
For e.g., for security level I, our balanced design achieves a speed-up of $4$-$5\times$ for both signing and verification times, whereas our high-performance design achieves a speed-up of $9$-$10\times$ for both signing and verification times when compared to the optimized software reference implementation. Notably, these gains are achieved while our design operates in the frequency range of \SI{115} MHz to \SI{132} MHz, whereas the optimized software implementation runs at $1.9$ GHz. 
However,
\ref{RQ:1} also asks about the resource cost
of accelerating \gls{MEDS}.
Since the computational cost
of \gls{MEDS} sign and verification
is significant,
selecting large performance parameters
for a high-performance design
results in high resource cost.

\medskip\noindent
\Cref{RQ:2} asks
to what extent resources can be shared
between the sign and verify operations
in a shared design
implementing both operations.
Since the main loop operates very similar
in both sign and verify,
the resources of sign
can be reused by verify
with only little control logic overhead.
Sign requires some additional computations
at the end
to compute the responses for the signature,
which can simply be skipped by verify.

\medskip\noindent
\Cref{RQ:3} asks
about the overhead of supporting all parameter sets
selectable at runtime
in a single joint design.
The overhead for additional control logic
for supporting different matrix dimensions
and challenge lengths
is marginal.
%
To our surprise,
supposing arithmetic in multiple prime fields
also comes at only a small overhead
as for the fields specified in \gls{MEDS},
most resources can be shared
between both fields.
Only slightly more logic
than needed for the larger field
is required to support both fields.
We conclude that
---
if the fields are chosen carefully
---
supporting multiple fields
in a joint hardware implementation
is very much feasible.
Specifically our results show
that arithmetic for
pseudo-Mersenne primes
with small Hamming distance
can be implemented jointly
with small overhead.



%
%\begingroup
%\makeatletter
%\def\@thefnmark{} \@footnotetext{
%\footnotesize}
%\endgroup
%%

\newcommand{\etalchar}[1]{$^{#1}$}
\begin{thebibliography}{CNP{\etalchar{+}}23b}

\bibitem[ABB{\etalchar{+}}24]{NISTPQC-ADD-R2:PERK24}
Najwa Aaraj, Slim Bettaieb, Lo\"{i}c Bidoux, Alessandro Budroni, Victor
  Dyseryn, Andre Esser, Thibauld Feneuil, Philippe Gaborit, Mukul Kulkarni,
  Victor Mateu, Marco Palumbi, Lucas Perin, Matthieu Rivain, {Jean-Pierre}
  Tillich, and Keita Xagawa.
\newblock {PERK}.
\newblock Technical report, {N}ational {I}nstitute of {S}tandards and
  {T}echnology, 2024.
\newblock available at
  \url{https://csrc.nist.gov/Projects/pqc-dig-sig/round-2-additional-signatures}.

\bibitem[ADKF70]{ArlazarovEtAl1970}
V.~Arlazarov, E.~Dinic, M.~Kronrod, and I.~Faradzev.
\newblock {On economical construction of the transitive closure of a directed
  graph}.
\newblock {\em Soviet Mathematics Doklady}, 11(5):1209--1210, 1970.

\bibitem[ALCZ20]{DBLP:conf/dsd/AmietLCZ20}
Dorian Amiet, Lukas Leuenberger, Andreas Curiger, and Paul Zbinden.
\newblock {FPGA}-based {SPHINCS}\({}^{\mbox{+}}\) {I}mplementations: Mind the
  {G}litch.
\newblock In {\em 23rd Euromicro Conference on Digital System Design, {DSD}
  2020, Kranj, Slovenia, August 26-28, 2020}, pages 229--237. {IEEE}, 2020.

\bibitem[AMI{\etalchar{+}}23]{9946370}
Aikata Aikata, Ahmet~Can Mert, Malik Imran, Samuel Pagliarini, and Sujoy~Sinha
  Roy.
\newblock {KaLi: A Crystal for Post-Quantum Security Using Kyber and
  Dilithium}.
\newblock {\em IEEE Transactions on Circuits and Systems I: Regular Papers},
  70(2):747--758, 2023.

\bibitem[BBB{\etalchar{+}}24a]{NISTPQC-ADD-R2:CROSS24}
Marco Baldi, Alessandro Barenghi, Michele Battagliola, Sebastian Bitzer, Marco
  Gianvecchio, Patrick Karl, Felice Manganiello, Alessio Pavoni, Gerardo
  Pelosi, Paolo Santini, Jonas Schupp, Edoardo Signorini, Freeman Slaughter,
  Antonia {Wachter-Zeh}, and Violetta Weger.
\newblock {{CROSS} --- {Codes and Restricted Objects Signature Scheme}}.
\newblock Technical report, {N}ational {I}nstitute of {S}tandards and
  {T}echnology, 2024.
\newblock available at
  \url{https://csrc.nist.gov/Projects/pqc-dig-sig/round-2-additional-signatures}.

\bibitem[BBB{\etalchar{+}}24b]{NISTPQC-ADD-R2:LESS24}
Marco Baldi, Alessandro Barenghi, Luke Beckwith, {Jean-Fran\c{c}ois} Biasse,
  Tung Chou, Andre Esser, Kris Gaj, Patrick Karl, Kamyar Mohajerani, Gerardo
  Pelosi, Edoardo Persichetti, {Markku-Juhani}~O. Saarinen, Paolo Santini,
  Robert Wallace, and Floyd Zweydinger.
\newblock {{LESS} --- {Linear Equivalence Signature Scheme}}.
\newblock Technical report, {N}ational {I}nstitute of {S}tandards and
  {T}echnology, 2024.
\newblock available at
  \url{https://csrc.nist.gov/Projects/pqc-dig-sig/round-2-additional-signatures}.

\bibitem[BCC{\etalchar{+}}24]{10.1007/978-3-031-54409-5_10}
Ward Beullens, Fabio Campos, Sofía Celi, Basil Hess, and Matthias~J.
  Kannwischer.
\newblock {Nibbling {MAYO}: Optimized Implementations for {AVX2} and
  {Cortex-M4}}.
\newblock {\em IACR Transactions on Cryptographic Hardware and Embedded
  Systems}, 2024(2):252–275, Mar. 2024.

\bibitem[BCH{\etalchar{+}}23]{TCHES:BCHKPSY23}
Ward Beullens, Ming-Shing Chen, Shih-Hao Hung, Matthias~J. Kannwischer, Bo-Yuan
  Peng, Cheng-Jhih Shih, and Bo-Yin Yang.
\newblock Oil and vinegar: Modern parameters and implementations.
\newblock {\em {IACR} {TCHES}}, 2023(3):321--365, 2023.

\bibitem[BFV13]{EC:BouFouVeb13}
Charles Bouillaguet, Pierre-Alain Fouque, and Amandine V{\'e}ber.
\newblock {Graph-Theoretic Algorithms for the ``Isomorphism of Polynomials''
  Problem}.
\newblock In Thomas Johansson and Phong~Q. Nguyen, editors, {\em
  EUROCRYPT~2013}, volume 7881 of {\em {LNCS}}, pages 211--227. Springer,
  Heidelberg, May 2013.

\bibitem[BNG21]{DBLP:conf/fpt/BeckwithNG21}
Luke Beckwith, Duc~Tri Nguyen, and Kris Gaj.
\newblock {High-Performance Hardware Implementation of CRYSTALS-Dilithium}.
\newblock In {\em International Conference on Field-Programmable Technology,
  {(IC)FPT} 2021, Auckland, New Zealand, December 6-10, 2021}, pages 1--10.
  {IEEE}, 2021.

\bibitem[CCD{\etalchar{+}}22]{TCHES:CCDLNSW22}
Po-Jen Chen, Tung Chou, Sanjay Deshpande, Norman Lahr, Ruben Niederhagen, Jakub
  Szefer, and Wen Wang.
\newblock {Complete and Improved {FPGA} Implementation of Classic {McEliece}}.
\newblock {\em {IACR} {TCHES}}, 2022(3):71--113, 2022.

\bibitem[CNP{\etalchar{+}}23a]{NISTPQC-ADD-R1:MEDS23}
Tung Chou, Ruben Niederhagen, Edoardo Persichetti, Lars Ran, Tovohery~Hajatiana
  Randrianarisoa, Krijn Reijnders, Simona Samardjiska, and Monika Trimoska.
\newblock {MEDS} --- {Matrix Equivalence Digital Signature}.
\newblock Technical report, {N}ational {I}nstitute of {S}tandards and
  {T}echnology, 2023.
\newblock available at
  \url{https://csrc.nist.gov/Projects/pqc-dig-sig/round-1-additional-signatures}.

\bibitem[CNP{\etalchar{+}}23b]{AFRICACRYPT:CNPRRST23}
Tung Chou, Ruben Niederhagen, Edoardo Persichetti, Tovohery~Hajatiana
  Randrianarisoa, Krijn Reijnders, Simona Samardjiska, and Monika Trimoska.
\newblock {Take Your {MEDS}: Digital Signatures from Matrix Code Equivalence}.
\newblock In Nadia {El Mrabet}, Luca {De Feo}, and Sylvain Duquesne, editors,
  {\em AFRICACRYPT 23}, volume 14064 of {\em {LNCS}}, pages 28--52. Springer
  Nature, July 2023.

\bibitem[CNRS24]{DBLP:conf/pqcrypto/ChouNRS24}
Tung Chou, Ruben Niederhagen, Lars Ran, and Simona Samardjiska.
\newblock {Reducing Signature Size of Matrix-Code-Based Signature Schemes}.
\newblock In Markku{-}Juhani~O. Saarinen and Daniel Smith{-}Tone, editors, {\em
  Post-Quantum Cryptography - 15th International Workshop, PQCrypto 2024,
  Oxford, UK, June 12-14, 2024, Proceedings, Part {I}}, volume 14771 of {\em
  Lecture Notes in Computer Science}, pages 107--134. Springer, 2024.

\bibitem[DHSY24]{DBLP:journals/tches/DeshpandeHSY24}
Sanjay Deshpande, James Howe, Jakub Szefer, and Dongze Yue.
\newblock {SDitH in Hardware}.
\newblock {\em {IACR} Trans. Cryptogr. Hardw. Embed. Syst.}, 2024(2):215--251,
  2024.

\bibitem[DLK{\etalchar{+}}25]{10.1145/3728469}
Sanjay Deshpande, Yongseok Lee, Cansu Karakuzu, Jakub Szefer, and Yunheung
  Paek.
\newblock {SPHINCSLET: An Area-Efficient Accelerator for the Full SPHINCS+
  Digital Signature Algorithm}.
\newblock {\em ACM Trans. Embed. Comput. Syst.}, April 2025.
\newblock Just Accepted.

\bibitem[dPRS23]{SP:dPRS23}
Rafa{\"e}l {del Pino}, Thomas Prest, M{\'e}lissa Rossi, and Markku-Juhani~O.
  Saarinen.
\newblock {High-Order Masking of Lattice Signatures in Quasilinear Time}.
\newblock In {\em 2023 {IEEE} Symposium on Security and Privacy}, pages
  1168--1185. {IEEE} Computer Society Press, May 2023.

\bibitem[FG18]{ferozpuri2018high}
Ahmed Ferozpuri and Kris Gaj.
\newblock {High-speed FPGA Implementation of the NIST Round 1 Rainbow Signature
  Scheme}.
\newblock In {\em 2018 International Conference on ReConFigurable Computing and
  FPGAs (ReConFig)}, pages 1--8, 2018.

\bibitem[FS87]{C:FiaSha86}
Amos Fiat and Adi Shamir.
\newblock {How to Prove Yourself: {Practical} Solutions to Identification and
  Signature Problems}.
\newblock In Andrew~M. Odlyzko, editor, {\em CRYPTO'86}, volume 263 of {\em
  {LNCS}}, pages 186--194. Springer, Heidelberg, August 1987.

\bibitem[GQT21]{DBLP:conf/stacs/GrochowQT21}
Joshua~A. Grochow, Youming Qiao, and Gang Tang.
\newblock {Average-Case Algorithms for Testing Isomorphism of Polynomials,
  Algebras}, and multilinear forms.
\newblock In Markus Bl{\"{a}}ser and Benjamin Monmege, editors, {\em 38th
  International Symposium on Theoretical Aspects of Computer Science, {STACS}
  2021, March 16-19, 2021, Saarbr{\"{u}}cken, Germany (Virtual Conference)},
  volume 187 of {\em LIPIcs}, pages 38:1--38:17. Schloss Dagstuhl -
  Leibniz-Zentrum f{\"{u}}r Informatik, 2021.

\bibitem[HGG07]{DBLP:conf/arith/HasenplaughGG07}
William Hasenplaugh, Gunnar Gaubatz, and Vinodh Gopal.
\newblock {Fast Modular Reduction}.
\newblock In {\em 18th {IEEE} Symposium on Computer Arithmetic {(ARITH-18}
  2007), 25-27 June 2007, Montpellier, France}, pages 225--229. {IEEE} Computer
  Society, 2007.

\bibitem[HQR89]{DBLP:journals/tc/HochetQR89}
Bertrand Hochet, Patrice Quinton, and Yves Robert.
\newblock {Systolic Gaussian Elimination over GF(p) with Partial Pivoting}.
\newblock {\em {IEEE} Trans. Computers}, 38(9):1321--1324, 1989.

\bibitem[HSK{\etalchar{+}}23]{cryptoeprint:2023/1267}
Florian Hirner, Michael Streibl, Florian Krieger, Ahmet~Can Mert, and
  Sujoy~Sinha Roy.
\newblock {Whipping the {MAYO} Signature Scheme using Hardware Platforms}.
\newblock Cryptology ePrint Archive, Paper 2023/1267, 2023.
\newblock \url{https://eprint.iacr.org/2023/1267}.

\bibitem[KKPY24]{cryptoeprint:2024/112}
Matthias~J. Kannwischer, Markus Krausz, Richard Petri, and Shang-Yi Yang.
\newblock {pqm4: Benchmarking {NIST} Additional Post-Quantum Signature Schemes
  on Microcontrollers}.
\newblock Cryptology {ePrint} Archive, Paper 2024/112, 2024.

\bibitem[LSG21]{DBLP:conf/cardis/LandSG21}
Georg Land, Pascal Sasdrich, and Tim G{\"{u}}neysu.
\newblock {A Hard Crystal - Implementing Dilithium on Reconfigurable Hardware}.
\newblock In Vincent Grosso and Thomas P{\"{o}}ppelmann, editors, {\em Smart
  Card Research and Advanced Applications - 20th International Conference,
  {CARDIS} 2021, L{\"{u}}beck, Germany, November 11-12, 2021, Revised Selected
  Papers}, volume 13173 of {\em Lecture Notes in Computer Science}, pages
  210--230. Springer, 2021.

\bibitem[MR24]{10483343}
Suraj Mandal and Debapriya~Basu Roy.
\newblock {KiD: A Hardware Design Framework Targeting Unified NTT
  Multiplication for CRYSTALS-Kyber and CRYSTALS-Dilithium on FPGA}.
\newblock In {\em 2024 37th International Conference on VLSI Design and 2024
  23rd International Conference on Embedded Systems (VLSID)}, pages 455--460,
  2024.

\bibitem[NQT24]{DBLP:conf/eurocrypt/NarayananQT24}
Anand~Kumar Narayanan, Youming Qiao, and Gang Tang.
\newblock {Algorithms for Matrix Code and Alternating Trilinear Form
  Equivalences via New Isomorphism Invariants}.
\newblock In Marc Joye and Gregor Leander, editors, {\em Advances in Cryptology
  - {EUROCRYPT} 2024 - 43rd Annual International Conference on the Theory and
  Applications of Cryptographic Techniques, Zurich, Switzerland, May 26-30,
  2024, Proceedings, Part {III}}, volume 14653 of {\em Lecture Notes in
  Computer Science}, pages 160--187. Springer, 2024.

\bibitem[Saa24]{DBLP:journals/iacr/Saarinen24}
Markku{-}Juhani~O. Saarinen.
\newblock {Accelerating {SLH-DSA} by Two Orders of Magnitude with a Single Hash
  Unit}.
\newblock {\em {IACR} Cryptol. ePrint Arch.}, page 367, 2024.
\newblock To appear in CRYPTO 2024, August, 2024.

\bibitem[SAW{\etalchar{+}}23]{DBLP:journals/iacr/SchmidAWZW23}
Michael Schmid, Dorian Amiet, Jan Wendler, Paul Zbinden, and Tao Wei.
\newblock {Falcon Takes Off - {A} Hardware Implementation of the Falcon
  Signature Scheme}.
\newblock {\em {IACR} Cryptol. ePrint Arch.}, page 1885, 2023.

\bibitem[SMA{\etalchar{+}}24]{DBLP:conf/cosade/SayariMAKS24}
Oussama Sayari, Soundes Marzougui, Thomas Aulbach, Juliane Kr{\"{a}}mer, and
  Jean{-}Pierre Seifert.
\newblock {HaMAYO: {A} Fault-Tolerant Reconfigurable Hardware Implementation of
  the {MAYO} Signature Scheme}.
\newblock In Romain Wacquez and Naofumi Homma, editors, {\em Constructive
  Side-Channel Analysis and Secure Design - 15th International Workshop,
  {COSADE} 2024, Gardanne, France, April 9-10, 2024, Proceedings}, volume 14595
  of {\em Lecture Notes in Computer Science}, pages 240--259. Springer, 2024.

\bibitem[SWM{\etalchar{+}}10]{DBLP:journals/tc/ShoufanWMHK10}
Abdulhadi Shoufan, Thorsten Wink, H.~Gregor Molter, Sorin~A. Huss, and Eike
  Kohnert.
\newblock {A Novel Cryptoprocessor Architecture for the McEliece Public-Key
  Cryptosystem}.
\newblock {\em {IEEE} Trans. Computers}, 59(11):1533--1546, 2010.

\bibitem[TDJ{\etalchar{+}}22]{EC:TDJPQS22}
Gang Tang, Dung~Hoang Duong, Antoine Joux, Thomas Plantard, Youming Qiao, and
  Willy Susilo.
\newblock Practical post-quantum signature schemes from isomorphism problems of
  trilinear forms.
\newblock In Orr Dunkelman and Stefan Dziembowski, editors, {\em
  EUROCRYPT~2022, Part~III}, volume 13277 of {\em {LNCS}}, pages 582--612.
  Springer, Heidelberg, May~/~June 2022.

\bibitem[TYD{\etalchar{+}}11]{tang2011high}
Shaohua Tang, Haibo Yi, Jintai Ding, Huan Chen, and Guomin Chen.
\newblock {High-Speed Hardware Implementation of Rainbow Signature on FPGAs}.
\newblock In Bo-Yin Yang, editor, {\em Post-Quantum Cryptography}, pages
  228--243, Berlin, Heidelberg, 2011. Springer Berlin Heidelberg.

\bibitem[WJW{\etalchar{+}}19]{SAC:WJWDGSN19}
Wen Wang, Bernhard Jungk, Julian W{\"a}lde, Shuwen Deng, Naina Gupta, Jakub
  Szefer, and Ruben Niederhagen.
\newblock {{XMSS} and Embedded Systems}.
\newblock In Kenneth~G. Paterson and Douglas Stebila, editors, {\em SAC 2019},
  volume 11959 of {\em {LNCS}}, pages 523--550. Springer, Heidelberg, August
  2019.

\bibitem[WSN16]{DBLP:conf/reconfig/WangSN16}
Wen Wang, Jakub Szefer, and Ruben Niederhagen.
\newblock {Solving large systems of linear equations over {GF(2)} on FPGAs}.
\newblock In Peter~M. Athanas, Ren{\'{e}} Cumplido, Claudia Feregrino, and Ron
  Sass, editors, {\em International Conference on ReConFigurable Computing and
  FPGAs, ReConFig 2016, Cancun, Mexico, November 30 - Dec. 2, 2016}, pages
  1--7. {IEEE}, 2016.

\bibitem[WSN17]{CHES:WanSzeNie17}
Wen Wang, Jakub Szefer, and Ruben Niederhagen.
\newblock {{FPGA}-based Key Generator for the Niederreiter Cryptosystem Using
  Binary Goppa Codes}.
\newblock In Wieland Fischer and Naofumi Homma, editors, {\em CHES~2017},
  volume 10529 of {\em {LNCS}}, pages 253--274. Springer, Heidelberg, September
  2017.

\bibitem[ZZC{\etalchar{+}}23]{DBLP:conf/dac/ZhuZCZLWL23}
Yihong Zhu, Wenping Zhu, Chen Chen, Min Zhu, Zhengdong Li, Shaojun Wei, and
  Leibo Liu.
\newblock {Mckeycutter: {A} High-throughput Key Generator of Classic McEliece
  on Hardware}.
\newblock In {\em 60th {ACM/IEEE} Design Automation Conference, {DAC} 2023, San
  Francisco, CA, USA, July 9-13, 2023}, pages 1--6. {IEEE}, 2023.

\bibitem[ZZL{\etalchar{+}}23]{zhu2023repqc}
Yihong Zhu, Wenping Zhu, Chongyang Li, Min Zhu, Chenchen Deng, Chen Chen,
  Shuying Yin, Shouyi Yin, Shaojun Wei, and Leibo Liu.
\newblock {RePQC: A 3.4-uJ/Op 48-kOPS Post-Quantum Crypto-Processor for
  Multiple-Mathematical Problems}.
\newblock {\em IEEE Journal of Solid-State Circuits}, 58(1):124--140, 2023.

\end{thebibliography}


\newpage
\appendix

\section*{Appendix}
\label{sec::appendix_algorithms}


\begin{algorithm}[H]
  \caption{MEDS sign (from \cite{NISTPQC-ADD-R1:MEDS23})}%\scname{}.\DEFSign(): signing}
  \label{alg:sign}
  %
  \BlankLine
  %
  \KwIn{secret key $\sk \in \B^{\ell_\sk}$,
        message $\mesg{} \in \B^{\ell_\mesg{}}$}
  \KwOut{signed message $\signedmessage{} \in \B^{\ell_\sig + \ell_{\mesg}}$}
  %
  \BlankLine
  $f_\sk \gets \lprivseed$\; \nllabel{alg::sign::start_init}
  \BlankLine
  $\sigma_{\mat{G}_0} \gets \sk[f_\sk, f_\sk + \lpubseed - 1]$\;
  $f_\sk \gets f_\sk + \lpubseed$\;
  \BlankLine
  $\mat{G}_0 \in \Fq{k \times mn} \gets \ExpandSystMat(\sigma_{\mat{G}_0})$\;
  \BlankLine
  \ForAll{$i \in \{1, \dots, s-1\}$}
  {
    $\mat{A}^{-1}_i \in \typeA \gets \decompress(\sk[f_\sk, f_\sk + \lmu], m, m)$\;
    $f_\sk \gets f_\sk + \lmu$\;
  }
  \BlankLine
  \ForAll{$i \in \{1, \dots, s-1\}$}
  {
    $\mat{B}^{-1}_i \in \typeB \gets \decompress(\sk[f_\sk, f_\sk + \lnu], n, n)$\;
    $f_\sk \gets f_\sk + \lnu$\; \nllabel{alg::sign::end_init}
  }
  \BlankLine
  %$\mat{G}_0 \in \typeG \gets (\mat{I}_k | \decompress(\sk[f_\sk,], k, mn-k) )$\;
  %$\mat{A}^{-1}_1, \dots \mat{A}^{-1}_{s-1} \in \typeA,
  %  \mat{B}^{-1}_1, \dots \mat{B}^{-1}_{s-1} \in \typeB,
  %  \mat{G}_0 \in \typeGcomp \gets \decompress(\sk[32,])$\;
  % \BlankLine
  % $\mat{G}_0 \in \typeG \gets (\mat{I}_k | \mat{G}_0)$\;
  \BlankLine
  %
  $\delta \in \B^{\lprivseed} \gets \randombytes(\lprivseed)$\; \nllabel{alg::sign::delta}
  %
  %$\sigma_0, \dots, \sigma_{t-1} \in \B^\lseed \gets \XOF(\delta, \lseed \times t)$\;
  $\rho \in \B^\lseed, \alpha \in \B^{\lsalt} \gets \XOF(\delta, \lseed, \lsalt)$\; \nllabel{alg::sign::root_salt}
  $\sigma_0, \dots, \sigma_{t-1} \in \B^\lseed \gets \SeedTree_t(\rho, \alpha)$\; \nllabel{alg::sign::seed_tree}
  %
  \BlankLine
  %
  \ForAll{$i \in \{0, \dots, t-1\}$ \nllabel{alg::sign::comm_start} }
  {
     %$\sigma'_i \in \B^\lseed \gets \sigma_i$\;
     \BlankLine
     $\sigma_i' \in \B^{\lsalt + \lseed + 4} \gets \left(\alpha | \sigma_i | \ToBytes(2^{1+\lceil\log_2(t)\rceil} + i, 4)\right)$\nllabel{alg::sign::startTildeABG}\;
     %$\sigma_{\mat{\tilde{A}}_i}, \sigma_{\mat{\tilde{B}}_i} \in \B^{32}; \sigma'_i \in \B^\lseed \gets 
     %       \XOF(\sigma_i, 32, 32, \lseed)$\nllabel{startTildeABG}\;
     $\sigma_{\mat{\tilde{A}}_i}, \sigma_{\mat{\tilde{B}}_i} \in \B^\lpubseed, \sigma_i \in \B^\lseed \gets 
            \XOF(\sigma_i', \lpubseed, \lpubseed, \lseed)$\; \nllabel{alg::sign::seed_xof}
     \BlankLine
     %
     $\mat{\tilde{A}}_i \in \GLmq
           \gets \ExpandInvMat(\sigma_{\mat{\tilde{A}}_i}, m)$\;
     $\mat{\tilde{B}}_i \in \GLnq
           \gets \ExpandInvMat(\sigma_{\mat{\tilde{B}}_i}, n)$\; \nllabel{alg::sign::gen_AB_end}
     \BlankLine
       $\mat{\tilde{G}}_i \in \typeG \gets
            \medspi{\mat{\tilde{A}}_i}{\mat{\tilde{B}}_i}{\mat{G}_0}$\; \nllabel{alg::sign::pi}
       $\mat{\tilde{G}}_i \in \typeG \cup \{\bot\} \gets \SF(\mat{\tilde{G}}_i)$\; \nllabel{alg::sign::SF}
     \BlankLine
     \If{$\mat{\tilde{G}}_i = \bot$\nllabel{alg::sign::is_syst}}
     {
      % $\sigma_i \in \B^\lseed \gets \sigma'_i$\;
       \KwGoto line \ref{alg::sign::startTildeABG}\;
     }
  } \nllabel{alg::sign::comm_end}
  \BlankLine
  %
  $d \in \B^\ldigest \gets \Hash(
     \compress(\mat{\tilde{G}}_0[;k,mn-1]) \,|\, \dots \,|\,
     \compress(\mat{\tilde{G}}_{t-1}[;k,mn-1])
     \, | \, \mesg)$\; \nllabel{alg::sign::comm_hash}
  $h_0, \dots, h_{t-1} \in \{0, \dots, s-1\} \gets \ParseHash(d)$\; \nllabel{alg::sign::parse_hash}
  \BlankLine
  %
  $f_v \gets 0$\;
  \BlankLine
  \ForAll{$i \in \{0, \dots, t-1\}$ \nllabel{alg::sign::mu_nu_start}}
  {
     \If{$h_i > 0$}
     {
       $\mu_i \in \typeA \gets \mat{\tilde{A}}_i \cdot \mat{A}^{-1}_{h_i}$\;
       $\nu_i \in \typeB \gets \mat{B}^{-1}_{h_i} \cdot \mat{\tilde{B}}_i$\;
       $v_{f_v} \in \B^{\lmu + \lnu}  \gets (\compress(\mu_i) \,|\, \compress(\nu_i))$\;
       $f_v \gets f_v + 1$\;
     }
     %{
     %  $\signedmessage{}_{,i} \in \B^\lseed \gets \sigma_i$\;
     %  %$\signedmessage{}_{,i} \in \B^\lseed \gets \bot$\;
     %}
     \nllabel{alg::sign::mu_nu_end}
  }
  \BlankLine
  %
  $p \in \B^{\ell_\spath} \gets \SeedTreeToPath_t(h_0, \dots, h_{t-1}, \rho, \alpha)$\; \nllabel{alg::sign::st2path}
  \BlankLine
  \Return $\signedmessage \in \B^{w(\lmu + \lnu) + \ell_\spath + \ldigest + \lsalt + \ell_\mesg{} = \ell_\sig + \ell_\mesg{}} = (
  %        \signedmessage{}_{,0} \,|\, \dots \,|\, \signedmessage{}_{,t-1} \,|\, d \,|\, m)$\;
          v_0 \,|\, \dots \,|\, v_{w-1} \,|\, p \,|\, d \,|\, \alpha \,|\, \mesg{})$\;
\end{algorithm}



\begin{algorithm}[p]
  \caption{MEDS verify (adapted from \cite{NISTPQC-ADD-R1:MEDS23})}%\scname{}.\DEFSign(): signing}
  \label{alg:verify}
  %
  \BlankLine
  %
  \KwIn{public key $\pk \in \B^{\ell_\pk}$,
        signed message $\signedmessage{} \in \B^{\ell_\sig + \ell_m}$}
  \KwOut{message $\mesg{} \in \B^\ell_m$ or $\bot$}
  \BlankLine
  %
  $\sigma_{\mat{G}_0} \gets \pk[0,\lpubseed-1]$\; \nllabel{alg::verify::start_init}
  $\mat{G}_0 \in \Fq{k \times mn} \gets \ExpandSystMat(\sigma_{\mat{G}_0})$\; \nllabel{alg::verify::init_G0}
  \BlankLine
  %
  %$\mat{\check{G}}_{1}, \dots, \mat{\check{G}}_{s-1} \gets \decompress(\pk[32,])$\;
  $f_\pk \gets \lpubseed$\;
  \BlankLine
  \ForAll{$i \in \{1, \dots, s-1\}$}
  {
    $\mat{G}_i \in \typeG \gets \decompressG(\pk[f_\pk,f_\pk + \ell_{G_i}])$\;
    $f_\pk \gets f_\pk + \ell_{G_i} $\; \nllabel{alg::verify::end_init}
    %$\mat{G}_i \in \typeG \gets (\mat{I}_k | \mat{\check{G}}_i)$\;
  }
  \BlankLine
  %
  $p \in \B^{\lpath}
     \gets
\signedmessage[\lsig-\ldigest-\lsalt-\lpath,\lsig-\ldigest-\lsalt-1]$\; \nllabel{alg::verify::start_parse_sig}
  $d \in \B^{\ldigest}, \alpha \in \B^{\lsalt}, \mesg \in \B^* \gets
           \signedmessage[\lsig-\ldigest-\lsalt,\lsig-\lsalt-1],
           \signedmessage[\lsig-\lsalt,\lsig-1],
           \signedmessage[\lsig,]$\; \nllabel{alg::verify::end_parse_sig}
  \BlankLine
  %
  $h_0, \dots, h_{t-1} \in \{0, \dots, s-1\} \gets \ParseHash(d)$\; \nllabel{alg::verify::parse_hash}
  \BlankLine
  $\sigma_0, \dots, \sigma_{t-1} \in \B^\lseed \gets \PathToSeedTree_t(h_0, \dots, h_{t-1}, p, \alpha)$\;
     \nllabel{alg:verify:path2st}
  \BlankLine
  %
  $f_{m_s} \gets 0$\;
  \BlankLine
  %
  \ForAll{$i \in \{0, \dots, t-1\}$ \nllabel{alg::verify::comm_start}}
  {
     \eIf{$h_i > 0$}
     {
       $\mu_i \in \typeA \gets \decompress(\signedmessage[f_{m_s},f_{m_s} + \lmu - 1], m, m)$\; \nllabel{alg::verify::mucopy}
       $\nu_i \in \typeB \gets \decompress(\signedmessage[f_{m_s} \! + \lmu,f_{m_s} \! \! + \lmu + \lnu - 1], n, n)$\; \nllabel{alg::verify::nucopy}
       $f_{m_s} \gets f_{m_s} + \lmu + \lnu$\;
       \BlankLine
       \If{$\mu_i \notin \typeAinv$ \KwOr $\nu_i \notin \typeBinv$}
       {
         \Return $\bot$\; \nllabel{alg::verify::munuexit}
       }
     }
     {
       % $\sigma_i \in \B^\lseed \gets \hat{\signedmessage}[f_{\hat{m}_s},f_{\hat{m}_s}+\lseed]$\;
       % $f_{\hat{m}_s} \gets f_{\hat{m}_s} + \lseed$\;
       % \BlankLine
       $\sigma_i' \in \B^{\lsalt + \lseed + 4} \gets \left(\alpha | \sigma_i | \nllabel{alg::verify::munustart} \ToBytes(2^{1+\lceil\log_2(t)\rceil} + i, 4)\right)$\;
       $\sigma_{\mat{\hat{A}}_i}, \sigma_{\mat{\hat{B}}_i} \in \B^\lpubseed, \sigma_i \in \B^\lseed \gets 
              \XOF(\sigma_i', \lpubseed, \lpubseed, \lseed)$\nllabel{startTildeABGver}\; \nllabel{alg::verify::seed_xof}
       \BlankLine
       %
       $\mu_i \in \typeAinv
             \gets \ExpandInvMat(\sigma_{\mat{\hat{A}}_i}, m)$\; 
       $\nu_i \in \typeBinv
             \gets \ExpandInvMat(\sigma_{\mat{\hat{B}}_i}, n)$\; \nllabel{alg::verify::munuend}
%       \BlankLine
%         $\mat{\hat{G}}_i \in \typeG \gets
%              \medspi{\mat{\hat{A}}_i}{\mat{\hat{B}}_i}{\mat{G}_0}$\;
%         $\mat{\hat{G}}_i \in \typeG \cup \{\bot\} \gets \SF(\mat{\hat{G}}_i)$\;
%       \BlankLine
%       \If{$\mat{\hat{G}}_i = \bot$}
%       {
%%         \Return $\bot$\;
%         %$\sigma_i \in \B^\lseed \gets \sigma'_i$\;
%         \KwGoto line \ref{startTildeABGver}\;
%       }
     }
     \BlankLine
     $\mat{\hat{G}}_i \in \typeG \gets
          \medspi{\mu_i}{\nu_i}{\mat{G}_{h_i}}$\; \nllabel{alg::verify::pi}
     $\mat{\hat{G}}_i \in \typeG \cup \{\bot\} \gets \SF(\mat{\hat{G}}_i)$\; \nllabel{alg::verify::SF}
     \BlankLine
     \If{$\mat{\hat{G}}_i = \bot$}
     {
       \Return $\bot$\;
     }
  } \nllabel{alg::verify::comm_end}
  \BlankLine
  %
  $d' \in \B^{\ldigest} \gets \Hash(
     \compress(\mat{\hat{G}}_0[;k,mn-1]) \,|\, \dots \,|\,
     \compress(\mat{\hat{G}}_{t-1}[;k,mn-1])
     \, | \, \mesg)$\; \nllabel{alg::verify::comm_hash}
  \BlankLine
  \eIf{$d = d'$}
  {
     \Return $\mesg$\; \nllabel{alg::verify::success}
  }
  {
     \Return $\bot$\; \nllabel{alg::verify::fail}
  }
\end{algorithm}




\end{document}

